\lab{Introduction to GitHub}{Introduction to GitHub}
\label{lab:GitHubIntro}
\objective{Git is a version control system that helps you manage changes to your code over time. It allows you to keep track of different versions of your code, collaborate with others, and revert changes if necessary. In ACME, Git will allow the Lab Assistants to see and grade your code. In this mini-lab you will learn how to successfully save your code to a GitHub repository.}

% Make tildes look nicer in the code boxes
\lstset{
    literate={~} {{\raisebox{.6ex}{\texttildelow}}}{1}
}

Before you begin this lab, you should have already gone through the Getting Started tutorials.
Specifically:
\begin{itemize}
\item The course materials should be downloaded and stored in an accessible place on your computer
\item VSCode (or another code editor) should be installed and set up on your computer
\item You should have created a GitHub account with repositories for Volume 1 and Volume 2
\item Python should be installed on your computer
\end{itemize}

If you have missed any of these steps, stop here and refer back to the Getting Started pdf and the accompanying tutorial videos.

Each week there will be an assigned lab in both Volume 1 and Volume 2 that will supplement the material presented in class.
These labs will involve editing, running, and saving code frequently.
To assist in this process, GitHub will be used to save your code and allow the instructors to grade each week.
The following example problem will help outline this process.

\begin{problem}
In the lab folder titled \li{GitHubIntro}, you will find the file \li{github_intro.py}.
Open \li{github_intro.py} with VSCode (or your favorite code editor).
In the function labeled \li{prob1()}, remove the line that says \li{raise NotImplementedError()} and replace it with \li{return "Student Name"} putting your actual name in the quotation marks.
Make sure to not change the indentation, otherwise your code will not run correctly.
Be sure to save the file after you've finished editing.

Now, run the file using python.
This can be done either in your code editor, or via the terminal (instructions are given later in this tutorial).
If everything worked correctly, you should see your name appear in the console.
\end{problem}

We will now demonstrate how to run your file in python and upload your changed file to GitHub.

To start, open a terminal window.
The following coding examples will teach you how to navigate to the \li{GitHubIntro/} directory via the terminal.
Instead of clicking icons to open folders, you will need to use typed commands within your terminal to move from folder to folder.
To see your current location, type the command \li{pwd} which stands for "print working directory."

\begin{lstlisting}
~$ pwd
/home/username
\end{lstlisting}

Use the command \li{ls} to list the contents of your current directory.

\begin{lstlisting}
~$ ls
Desktop		Downloads		Public 		Videos
Documents 	Pictures
\end{lstlisting}

Use the command \li{cd} to change directory. For example, if I wanted to navigate into the \li{Documents} folder, I would use the following command:

\begin{lstlisting}
~$ cd Documents				# Change to the Documents folder
~$ pwd						# Check that the current directory changed
/home/username/Documents
\end{lstlisting}

If you wish to go back a directory, use \li{cd} followed by a space and two periods ``{..}''

\begin{lstlisting}
~$ cd ..
~$ pwd
/home/username
\end{lstlisting}

Using these commands, navigate to the \li{GitHubIntro} directory.
Once you're there, use \li{pwd} to check that you're in the right folder.
It should look like the following example.

\begin{lstlisting}
~$ pwd
/home/username/.../GitHubIntro
\end{lstlisting}

Now that you are in the correct directory, you can run the following command to see that your code compiles correctly:

\begin{lstlisting}
~$ python github_intro.py
Student Name
\end{lstlisting}

Now, load your changed file to GitHub using the following commands:

\begin{lstlisting}[language=bash]
# Pull changes from your online repository to your local machine
~$ git pull origin master

# Tell git to track changes made to github_intro.py
~$ git add github_intro.py

# Commit saves the changes you made to your file, acting like a time stamp
~$ git commit -m "Finished GitHub lab"

# Send your commits to the repository
~$ git push origin master
\end{lstlisting}

Now, open a browser and log in to your GitHub account.
Open your Volume 1 repository and open your \li{github_intro.py} file.
If everything worked correctly, you should be able to see the edits you made to your file.
Additionally, you can click on \li{Commits} in the menu on the left and see your most recent commit statements.
Congratulations, you have completed your GitHub introduction!

In the future, you will submit your labs in a similar way.
The process is to first navigate to the correct directory within your terminal, then input the following commands:

\begin{lstlisting}[language=bash]
~$ git pull origin master
~$ git add <changed files>
~$ git commit -m "<descriptive message>"
~$ git push origin master
\end{lstlisting}

Note: following this ordering will help you to avoid merge conflicts with GitHub!
