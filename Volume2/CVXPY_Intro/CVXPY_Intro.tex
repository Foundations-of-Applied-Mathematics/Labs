\lab{CVXPY}{CVXPY}
\objective{CVXPY is a package of Python functions and classes designed for the purpose of convex optimization.
In this lab we use these tools for linear and quadratic programming.
We will solve various optimization problems using CVXPY and optimize eating healthily on a budget.}

\section*{Linear Programs} % ==================================================

A \emph{linear program} is a linear constrained optimization problem. Such a problem can be stated in several
different forms, one of which is
\begin{align*}
\text{minimize}\qquad &\c\trp \x \\
\text{subject to}\qquad &G\x \preceq \mathbf{h}\\
&A\x = \b.
\end{align*}

The symbol $\preceq$ denotes that the components of $G\x$ are less than the components of $\mathbf{h}$.
In other words, if $\x\preceq\y$, then $x_i < y_i$ for all $x_i\in\x$ and $y_i\in\y$. 
CVXPY accepts  $\leq, \geq$, and $=$ in its constraints as long as the equations satisfy convexity requirements described later in this chapter, so we can reformulate this problem in yet another form:
\begin{align*}
\text{minimize}\qquad &\c\trp \x \\
\text{subject to}\qquad &G\x \preceq \mathbf{h}\\
&P\x \succeq \mathbf{q}\\
&A\x = \b.
\end{align*}
CVXPY accepts NumPy arrays and SciPy sparse matrices for the constraints, but the variable $\x$ must be a CVXPY \li{Variable}.

% Students have not yet learned about the dual problem. May be included in a later lab.
\begin{comment}
The corresponding \emph{dual program} for the above linear program has the form
\begin{align*}
\text{maximize}\qquad &-h\trp z - b\trp y \\
\text{subject to}\qquad &G\trp z + A\trp y + c = 0\\
 &z \geq 0.
\end{align*}
CVXOPT provides functions to solve both the original (\emph{primal}) linear program and its dual program.
\end{comment}

Consider the following example:
\begin{align*}
\text{minimize}\qquad &-4x_1-5x_2 \\
\text{subject to}\qquad &x_1+2x_2 \leq 3 \\
	        &2x_1+x_2 = 3 \\
		&x_1, x_2 \geq 0
\end{align*}
We can solve this problem using the following code.
Note that \li{cvxpy.Problem()} accepts the constraints as a single list and that \li{>=} represents both standard and elementwise greater than or equal to.
The symbols \li{<=} and \li{==} are similarly versatile.

\begin{lstlisting}
>>> import cvxpy as cp
>>> import numpy as np

# First we'll initialize the objective
# We can declare x with its size and sign
# nonneg = True is equivalent to the constraint P@ x >= 0 listed below
# Both are included for demonstration but are redundant
>>> x = cp.Variable(2, nonneg = True)	
>>> c = np.array([-4, -5])
>>> objective = cp.Minimize(c.T @ x)

# Then we'll write the constraints
>>> A = np.array([2, 1])
>>> G = np.array([1, 2])
>>> P = np.eye(2)
>>> constraints = [A @ x == 3, G @ x <= 3, P @ x >= 0] #This must be a list

# Assemble the problem and then solve it
>>> problem = cp.Problem(objective, constraints)
>>> print(problem.solve())
-8.999999999850528
>>> print(x.value)
array([1., 1.])
\end{lstlisting}
%CVXPY can take in == or >= or <= so I deleted an irrelevant section here about switching constraint types

%CVXPY Install Problem
\begin{warn}
	If you are having trouble with \li{pip install cvxpy} or \li{conda install cvxpy} check the following:
	\begin{itemize}
		\item CVXPY requires a C++ compiler, most MacOs ad Linux Systems have them built in. 
		If you are running Windows, make sure that you have the "C++ builder tools" from the Visual Studio Build Tools installed.
		\item CVXPY requires specific versions of packages in order to run, check that you have the right version of your packages. 
		The most common is NumPy is not up to date.
	\end{itemize}
\end{warn}

%Problem 1
\begin{problem}
Solve the following convex optimization problem:
\begin{align*}
\text{minimize}\qquad &2x_1+x_2+3x_3 \\
\text{subject to}\qquad &x_1+2x_2 \leq 3 \\
		&x_2-4x_3\leq 1\\
	        &2x_1+10x_2+3x_3 \geq 12 \\
		&x_1 \geq 0 \\
		&x_2 \geq 0 \\
		&x_3 \geq 0
\end{align*}
Return the minimizer $\x$ and the primal objective value.
\end{problem}
\subsection*{$l_1$ Norm}
The $l_1$ norm is defined 
\[||\x||_1=\sum_{i=1}^n |x_i|.\]
An $l_1$ minimization problem is minimizing a vector's $l_1$ norm, while fitting certain constraints. It can be written in the following form:
\begin{align*}
\text{minimize}\qquad &\|\x\|_1\\
\text{subject to} \qquad &A\x = \b.
\end{align*}

CVXPY includes the $l_1$ norm and many other useful functions.
To specify a norm in CVXPY, use the syntax \li{cp.norm(x, a)} where $a$ represents your choice of norm (1 in this case).
%Removed a big section here about how to modify CVXOPT to do the L1 norm. It's significantly trickier than with CVXPY.

%Problem 2
\begin{problem}
Write a function called \li{l1Min()} that accepts a matrix $A$ and vector $\mathbf{b}$ as NumPy arrays and solves the $l_1$ minimization problem.
Return the minimizer $\x$ and the primal objective value.

To test your function consider the matrix $A$ and vector $\mathbf{b}$ below.
\[
A = \begin{bmatrix}
1 & 2 & 1 & 1\\
0 & 3 & -2 & -1
\end{bmatrix} \qquad
\mathbf{b} = \begin{bmatrix}
7 \\
4
\end{bmatrix}
\]
The linear system $A\x = \b$ has infinitely many solutions.
Use \li{l1Min()} to verify that the solution which minimizes $||\mathbf{x}||_1$ is approximately $\x = [0., 2.571, 1.857, 0.]^T$ and the minimum objective value is approximately $4.429$.
There's a file called \li{test_cvxpy_intro.py} that contains this example as a prewritten unit test that you can use to test your function for this problem.
\label{prob:l1}
\end{problem}

\section*{The Transportation Problem}

Consider the following transportation problem:
A piano company needs to transport thirteen pianos from their three  supply centers (denoted by 1, 2, 3) to two demand centers (4, 5).
Transporting a piano from a supply center to a demand center incurs a cost, listed in Table \ref{tab:cost}.
The company wants to minimize shipping costs for the pianos while meeting the demand.
%How many pianos should each supply center send to each demand center?

\begin{table}[H]
\centering
\begin{tabular}{|c|c|}
Supply Center & Number of pianos available\\
\hline
1 & 7\\
2 & 2\\
3 & 4\\
\end{tabular}

\caption{Number of pianos available at each supply center}
\label{tab:supply}
\end{table}

\begin{table}[H]
\centering
\begin{tabular}{|c|c|}
Demand Center & Number of pianos needed\\
\hline
4 & 5\\
5 & 8\\
\end{tabular}

\caption{Number of pianos needed at each demand center}
\label{tab:demand}
\end{table}

\begin{table}[H]
\centering
\begin{tabular}{|c|c|c|c|}
Supply Center & Demand Center & Cost of transportation & Number of pianos\\
\hline
1 & 4 & 4 & $p_1$\\
1 & 5 & 7 & $p_2$\\
2 & 4 & 6 & $p_3$\\
2 & 5 & 8 & $p_4$\\
3 & 4 & 8 & $p_5$\\
3 & 5 & 9 & $p_6$\\
\end{tabular}
\caption{Cost of transporting one piano from a supply center to a demand center}
\label{tab:cost}
\end{table}

A system of constraints can be defined using the variables $p_1,p_2,p_3,p_4,p_5,$ and $p_6$.
First, there cannot be a negative number of pianos transported along any route.
Next, use tables \ref{tab:supply} and \ref{tab:demand} and the variables $p_1...p_6$ to define a supply or demand constraint for each location.
You may want to format this as a matrix.
\begin{comment} %I think they can come up with these on their own.
the following three supply constraints and two demand constraints:
\begin{align*}
p_1 + p_2  &= 7\\
p_3 + p_4  &= 2\\
p_5 + p_6  &= 4\\
p_1 + p_3 + p_5 &= 5\\
p_2 + p_4 + p_6 &= 8
\end{align*}
\end{comment}
Finally, the objective function is the number of pianos shipped along each route multiplied by the respective costs (Table \ref{tab:cost}).
%\[
%4p_1 + 7p_2 + 6p_3 + 8p_4 + 8p_5 + 9p_6.
%\]

\begin{info}
Since our answers must be integers, in general this problem turns out to be an NP-hard problem.
There is a whole field devoted to dealing with integer constraints, called \emph{integer linear programming}, which is beyond the scope of this lab.
Fortunately, we can treat this particular problem as a standard linear program and still obtain integer solutions.
\end{info}

%Problem 3. The point was originally to change equality constraints to inequality, but that's not necessary with CVXPY, so now it's mostly about interpreting the tables.
\begin{problem}
Solve the piano transportation problem. %by converting the last equality constraint into an inequality constraint.
Return the minimizer $\x$ and the primal objective value.
\end{problem}

\section*{Quadratic Programming}

Quadratic programming is similar to linear programming, but the objective function is quadratic rather than linear.
The constraints, if there are any, are still of the same form.
Thus, $G, \mathbf{h}, A$, and $\b$ are optional.
The formulation that we will use is
\begin{align*}
\text{minimize}\qquad &\frac{1}{2}\x\trp Q\x + \mathbf{r}\trp \x \\
\text{subject to}\qquad &G\x \preceq \mathbf{h}\\
 &A\x = \b,
\end{align*}
where $Q$ is a positive semidefinite symmetric matrix.
%In this formulation, we require again that $A$ has full row rank and that the block matrix
%$[Q \quad G \quad A]\trp $ has full column rank.

As an example, consider the quadratic function
\[
f(x_1,x_2) = 2x_1^2 +2x_1x_2 + x_2^2 +x_1 -x_2.
\]
There are no constraints, so we only need to initialize the matrix $Q$ and the vector $\mathbf{r}$.
To find these, we first rewrite our function to match the formulation given above.
If we let
\[
Q = \begin{bmatrix}
  a & b\\
  b & c
\end{bmatrix}, \qquad
\mathbf{r} = \begin{bmatrix}
  d\\
  e
\end{bmatrix},
\qquad \text{and} \qquad
\x = \begin{bmatrix}
  x_1\\
  x_2
\end{bmatrix},
\]
then
\begin{align*}
\frac{1}{2}\x\trp Q\x + \mathbf{r}\trp \x &=
\frac{1}{2}
\begin{bmatrix}
  x_1\\
  x_2
\end{bmatrix}\trp
\begin{bmatrix}
  a & b\\
  b & c
\end{bmatrix}
\begin{bmatrix}
  x_1\\
  x_2
\end{bmatrix} +
\begin{bmatrix}
  d\\
  e
\end{bmatrix}\trp
\begin{bmatrix}
  x_1\\
  x_2
\end{bmatrix} \\
&= \frac{1}{2}ax_1^2 + bx_1x_2 + \frac{1}{2}cx_2^2 + dx_1 + ex_2
\end{align*}
Thus, we see that the proper values to initialize our matrix $Q$ and vector $\mathbf{r}$ are:
\begin{align*}
a &= 4  &d = 1 \\
b &= 2  &e = -1 \\
c &= 2
\end{align*}
Now that we have the matrix $Q$ and vector $\mathbf{r}$, we are ready to use the CVXPY function for quadratic programming, \li{cp.quad_form()}.
\begin{lstlisting}
>>> Q = np.array([[4, 2],[2, 2]])
>>> r = np.array([1, -1])
>>> x = cp.Variable(2)
>>> prob = cp.Problem(cp.Minimize(.5 * cp.quad_form(x, Q) + r.T @ x))
>>> print(prob.solve())
[-1. 1.5]
>>> print(x.value)
-1.25
\end{lstlisting}

%Problem 4
\begin{problem}
Find the minimizer and minimum of
\begin{equation*}
g(x_1,x_2,x_3) = \frac{3}{2}x_1^2 +2x_1x_2 + x_1x_3+ 2x_2^2 +2x_2x_3+\frac{3}{2}x_3^2+3x_1 + x_3
\end{equation*}
\\(Hint: Write the function $g$ to match the formulation given above before coding.)
\begin{comment}
\begin{equation}
f(x) = \frac{1}{2}x\trp Qx - x\trp p
\end{equation}
where

\begin{center}
$Q =
\begin{bmatrix}
3 & 2 & 1\\
2 & 4 & 2\\
1 & 2 & 3\\
\end{bmatrix}
$
and $p =
\begin{bmatrix}
3\\
0\\
1\\
\end{bmatrix}
$
\end{center}
\end{comment}

\end{problem}

So far we have only dealt with affine constraints.
When working with non-affine constraints, be aware that CVXPY comes with some Disciplined Convex Programming (DCP) rules.
A minimization problem requires a convex objective function; similarly, a maximization problem requires a concave objective function.
Equality constraints (\li{==}) must be affine.
Less-than constraints (\li{<=}) must have the left side convex and the right side concave.
Greater-than constraints (\li{>=}) must have the left side concave and the right side convex.
This webpage provides a list of which of CVXPY's functions are concave or convex.
\url{https://www.cvxpy.org/tutorial/functions/index.html}

%Problem 5
\begin{problem}
Write a function that accepts  a matrix $A$ and vector $\b$ and solves the following problem.
\begin{align*}
\text{minimize}\qquad &\|A\x - \b\|_2\\
\text{subject to}\qquad &\|\x\|_1 = 1\\
		&\x \succeq 0\\
\end{align*}
To test your function, use the matrix $A$ and vector $\b$ from Problem \ref{prob:l1}. 
The minimizer is approximately $\x=[0, 1, 0, 0]$ with objective value $5.099$.
Hint: \li{norm()} is a convex function, so you will have to think of a different way to take the 1-norm.
\end{problem}

\begin{unittest}
There is a file called \li{test_cvxpy_intro.py} that contains prewritten unit tests for Problem 2. 
There is a place for you to add your own unit tests to test your function for Problem 5, which will be graded.
\end{unittest}

\begin{comment}
\begin{problem} %Given that CVXPY comes with built-in norm functions, this feels too easy or maybe out of place.
The $l_2$ maximization problem is to
\begin{align*}
\text{maximize}\qquad &\|\x\|_2\\
\text{subject to} \qquad &A\x = \b.
\end{align*}

This problem is equivalent to a quadratic program, since $\|\x\|_2 = \x\trp \x$.
Write a function that accepts a matrix $A$ and vector $\b$ and solves the $l_2$ maximization problem.
Return the maximizer $\x$ and the primal objective value.

To test your function, use the matrix $A$ and vector $\b$ from Problem \ref{prob:l1}. 
The minimizer is approximately $\x=[0.966, 2.169, 0.809, 0.888]^T$ and the minimum primal objective value is approximately $7.079$.
\end{problem}
\end{comment}
\begin{comment}
\section*{Allocation Models}
Allocation models lead to simple linear programs. An allocation model seeks to allocate a valuable resource among competing needs. Consider the following example taken from ``Optimization in Operations Research" by Ronald L. Rardin. %%pg 132

The U.S. Forest service has used an allocation model to deal with the task of managing national forests.
The model begins by dividing the land into a set of analysis areas. Several land management policies (also
called prescriptions) are then proposed and evaluated for each area.
An \emph{allocation} is how much land (in acreage) in each unique analysis area will be assigned to each of the possible prescriptions.
We seek to find the best possible allocation, subject to forest-wide restrictions on land use.

The file \li{ForestData.npy} contains data for a fictional national forest (you can also find the data
in Table \ref{tab:forest}). There are 7 areas of analysis and 3 prescriptions for each of them.

\begin{align*}
&\text{Column 1: $i$, area of analysis} \\
&\text{Column 2: $s_i$, size of the analysis area (in thousands of acres)} \\
&\text{Column 3: $j$, prescription number} \\
&\text{Column 4: $p_{i,j}$, net present value (NPV) per acre in area $i$ under prescription $j$} \\
&\text{Column 5: $t_{i,j}$, protected timber yield per acre in area $i$ under prescription $j$} \\
&\text{Column 6: $g_{i,j}$, protected animal grazing capability per acre for area $i$ under prescription $j$} \\
&\text{Column 7: $w_{i,j}$, wilderness index rating (0 to 100) for area $i$ under prescription $j$}
\end{align*}

\begin{table}[H]
\centering
    \begin{tabular}{c c c c c c c}
&&&Forest Data&&& \\
\hline
Analysis & Acres &Prescrip-&NPV&Timber&Grazing&Wilderness \\
Area&(1000)'s &tion&(per acre) &(per acre)&(per acre)& Index\\
$i$ &$s_i$&$j$& $p_{i,j}$ & $t_{i,j}$&$g_{i,j}$&$w_{i,j}$ \\\hline
1&	75	&1	&503	&310	&0.01&	40\\
&&		2&	140&	50&	0.04	&80\\
&&		3&	203&	0&	0&	95\\ \hline
2&	90&	1	&675&	198&	0.03&	55\\
&&		2&	100&	46&	0.06&	60\\
&&		3&	45&	0&	0&	65\\ \hline
3&	140&	1	&630&	210	&0.04&	45\\
&&		2&	105&	57&	0.07&	55\\
&&		3&	40	&0&	0&	60\\ \hline
4	&60&	1&	330&	112&	0.01&	30\\
&&		2	&40&	30&	0.02&	35\\
&&		3&	295&	0&	0	&90\\ \hline
5	&212&	1	&105	&40	&0.05&	60\\
&&		2	&460&	32	&0.08&	60\\
&& 3	&120&0&	0	&70\\ \hline
6	&98	&1	&490	&105	&0.02	&35\\
&&		2&	55	&25	&0.03	&50\\
&&		3	&180	&0	&0	&75\\ \hline
7&	113&	1	&705	&213&	0.02	&40\\
&&		2&	60	&40	&0.04&	45\\
&&		3	&400	&0	&0	&95\\
\hline
    \end{tabular}
\caption{}
\label{tab:forest}
\end{table}
Let $x_{i,j}$ be the amount of land in area $i$ allocated to prescription $j$.
Under this notation, an allocation is a one-dimensional vector consisting of the $x_{i,j}$'s. 
For this particular
example, there are 7 acres, with 3 prescriptions each.
So the allocation vector is a one-dimensional vector with 21 entries.
Our goal is to find the allocation vector that maximizes net present value, while producing at least 40 million
board-feet of timber, at least 5 thousand units of grazing capability, and keeping the average wilderness index at least 70.
The allocation vector is also constrained to be nonnegative, and all of the land must be allocated precisely.

Since acres are in thousands, divide the constraints of timber and animal grazing by 1000 in the problem setup, and compensate for this after obtaining a solution.

The problem can be written as follows:
\begin{align*}
\text{maximize } &\sum\limits_{i=1}^7 \sum\limits_{j=1}^3 p_{i,j}x_{i,j} \\
\text{subject to } &\sum\limits_{j=1}^3 x_{i,j} = s_i  \text{ for } i=1,..,7 \\
	        &\sum\limits_{i=1}^7 \sum\limits_{j=1}^3 t_{i,j}x_{i,j} \geq 40,000 \\
		&\sum\limits_{i=1}^7 \sum\limits_{j=1}^3 g_{i,j}x_{i,j} \geq 5 \\
		&\frac{1}{788} \sum\limits_{i=1}^7 \sum\limits_{j=1}^3 w_{i,j}x_{i,j} \geq 70 \\
		&x_{i,j} \geq 0 \text{ for } i=1,...,7  \text{ and } j=1,2,3
\end{align*}

\begin{problem}
Solve the allocation problem above.
Return the minimizing allocation vector of $x_{i,j}$'s and the maximum total net present value.
Remember to consider the following:
\begin{enumerate}
\item The allocation vector should be a (21,1) NumPy array.
\item Recall that the constraints of timber and animal grazing were divided by 1000.
To compensate, the maximum total net value will be equal to the primal objective of the appropriately minimized linear function multiplied by -1000.
\end{enumerate}
\end{problem}

\end{comment}

\section*{Eating on a Budget}

In 2009, the inmates of Morgan County jail convinced Judge Clemon of the Federal District Court in Birmingham to put Sheriff Barlett in jail for malnutrition.
Under Alabama law, in order to encourage less spending, "the chief lawman could go light on prisoners' meals and pocket the leftover change."\footnote[1]{Nossiter, Adam, 8 Jan 2009, "As His Inmates Grew Thinner, a Sheriff’s Wallet Grew Fatter", \emph{New York Times},\url{https://www.nytimes.com/2009/01/09/us/09sheriff.html}}.
Sheriffs had to ensure a minimum amount of nutrition for inmates, but minimizing costs meant more money for the sheriffs themselves.
Judge Clemon jailed Sheriff Barlett until a plan was made to use all allotted funds, \$1.75 per inmate, to feed prisoners more nutritious meals.
While this case made national news, the controversy of feeding prisoners in Alabama continues as of 2019\footnote[2]{Sheets, Connor, 31 January 2019, "Alabama sheriffs urge lawmakers to get them out of the jail food business", \url{https://www.al.com/news/2019/01/alabama-sheriffs-urge-lawmakers-to-get-them-out-of-the-jail-food-business.html}}.

The problem of minimizing cost while reaching healthy nutritional requirements can be approached as a convex optimization problem.
Rather than viewing this problem from the sheriff's perspective, we view it from the perspective of a college student trying to minimize food cost in order to pay for higher education, all while meeting standard nutritional guidelines.

The file \li{food.npy} contains a dataset with nutritional facts for 18 foods that have been eaten frequently by college students working on this text.
A subset of this dataset can be found in Table \ref{tab:food-data}, where the "Food" column contains the list of all 18 foods.

The columns of the full dataset are:
\begin{align*}
& \text{Column 1: $p$, price (dollars)} \\
& \text{Column 2: $s$, servings per container} \\
& \text{Column 3: $c$, calories per serving} \\
& \text{Column 4: $f$, fat per serving (grams)} \\
& \text{Column 5: $\hat{s}$, sugar per serving (grams)} \\
& \text{Column 6: $\hat{c}$, calcium per serving (milligrams)} \\
& \text{Column 7: $\hat{f}$, fiber per serving (grams)} \\
& \text{Column 8: $\hat{p}$, protein per serving (grams)}
\end{align*}

 \begin{table}[H]
% \begin{adjustwidth}{-.5in}{-.5in}
\begin{tabular}{|c||c|c|c|c|c|c|c|c|c|c|}
\hline
\textbf{Food} & \textbf{Price} & \textbf{Servings} & \textbf{Calories} & \textbf{Fat} & \textbf{Sugar} & \textbf{Calcium} & \textbf{Fiber} & \textbf{Protein} \\ 
& $p$ & $s$ & $c$ & $f$ & $\hat{s}$ & $\hat{c}$ & $\hat{f}$ & $\hat{p}$ \\ 
& dollars & & & g & g & mg & g & g \\ \hline\hline
Ramen & 6.88 & 48 & 190 & 7 & 0 & 0 & 0 & 5 \\ \hline
Potatoes & 0.48 & 1 & 290 & 0.4 & 3.2 & 53.8 & 6.9 & 7.9 \\ \hline
Milk & 1.79 & 16 & 130 & 5 & 12 & 250 & 0 & 8 \\ \hline
Eggs & 1.32 & 12 & 70 & 5 & 0 & 28 & 0 & 6 \\ \hline
Pasta & 3.88 & 8 & 200 & 1 & 2 & 0 & 2 & 7 \\ \hline
Frozen Pizza & 2.78 & 5 & 350 & 11 & 5 & 150 & 2 & 14 \\ \hline
Potato Chips & 2.12 & 14 & 160 & 11 & 1 & 0 & 1 & 1 \\ \hline
Frozen Broccoli & 0.98 & 4 & 25 & 0 & 1 & 25 & 2 & 1 \\ \hline
Carrots & 0.98 & 2 & 52.5 & 0.3 & 6.1 & 42.2 & 3.6 & 1.2 \\ \hline
Bananas & 0.24 & 1 & 105 & 0.4 & 14.4 & 5.9 & 3.1 & 1.3 \\ \hline
Tortillas & 3.48 & 18 & 140 & 4 & 0 & 0 & 0 & 3 \\ \hline
Cheese & 1.88 & 8 & 110 & 8 & 0 & 191 & 0 & 6 \\ \hline
Yogurt & 3.47 & 5 & 90 & 0 & 7 & 190 & 0 & 17 \\ \hline
Bread & 1.28 & 6 & 120 & 2 & 2 & 60 & 0.01 & 4 \\ \hline
Chicken & 9.76 & 20 & 110 & 3 & 0 & 0 & 0 & 20 \\ \hline
Rice & 8.43 & 40 & 205 & 0.4 & 0.1 & 15.8 & 0.6 & 4.2 \\ \hline
Pasta Sauce & 3.57 & 15 & 60 & 1.5 & 7 & 20 & 2 & 2 \\ \hline
Lettuce & 1.78 & 6 & 8 & 0.1 & 0.6 & 15.5 & 1 & 0.6 \\ \hline
\end{tabular}
\caption{Subset of table containing food data}
\label{tab:food-data}
% \end{adjustwidth}
\end{table}

 According to the FDA\footnote[1]{url{https://www.accessdata.fda.gov/scripts/InteractiveNutritionFactsLabel/pdv.html}} and US Department of Health, someone on a $2000$ calorie diet should have no more than 2000 calories, no more than 65 grams of fat, no more than 50 grams of sugar\footnote[2]{https://www.today.com/health/4-rules-added-sugars-how-calculate-your-daily-limit-t34731}, at least 1000 milligrams of calcium\footnote[1]{26 Sept 2018, \url{https://ods.od.nih.gov/factsheets/Calcium-HealthProfessional/}}, at least 25 grams of fiber, and at least 46 grams of protein\footnote[2]{\url{https://www.accessdata.fda.gov/scripts/InteractiveNutritionFactsLabel/protein.html}} per day.

 We can rewrite this as a convex optimization problem below.

 \begin{align*}
\text{minimize } & \sum_{i=1}^{18}p_ix_i, \\
\text{subject to }& \sum_{i=1}^{18} c_ix_i \leq 2000, \\
			& \sum_{i=1}^{18} f_ix_i \leq 65, \\
			& \sum_{i=1}^{18} \hat{s}_ix_i \leq 50, \\
			& \sum_{i=1}^{18} \hat{c}_ix_i \geq 1000, \\
			& \sum_{i=1}^{18} \hat{f}_ix_i \geq 25, \\
			& \sum_{i=1}^{18} \hat{p}_ix_i \geq 46, \\
			& x_i \geq 0.
\end{align*}

%Problem 6
 \begin{problem}
Read in the file \li{food.npy}.
Use CVXPY to identify how much of each food item a college student should each to minimize cost spent each day given these simplified nutrition requirements.
Return the minimizing vector and the total amount of money spent.

According to this problem, what is the food you should eat most each day? 
What are the three foods you should eat most each week?

(Hint: Each nutritional value must be multiplied by the number of servings to get the nutrition value of the whole product).
\label{prob:diet}
\end{problem}

You can learn more about CVXPY at
\url{https://www.cvxpy.org/index.html}.
