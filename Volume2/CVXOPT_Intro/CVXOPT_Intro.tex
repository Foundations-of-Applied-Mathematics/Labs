\lab{CVXOPT}{CVXOPT}
\objective{CVXOPT is a package of Python functions and classes designed for the purpose of convex optimization.
In this lab we use these tools for linear and quadratic programming.
We will solve various optimization problems using CVXOPT and optimize allocating land using linear programming.}

%\begin{warn}
%CVXOPT is not part of the standard library, and it is only included in the Anaconda distribution for Python 3.6 for Linux and Mac.
%We recommend avoiding Windows machines for this lab.
%
%To install CVXOPT, use \li{conda install cvxopt} or \li{pip install cvxopt}.
%\end{warn}

\section*{Linear Programs} % ==================================================

%%Cvxopt has linear program solver and can implement integer programming through the Gnu Linear Programming Kit, glpk.
%CVXOPT is a package of Python functions and classes designed for the purpose of convex optimization.
%In this lab we will focus on linear and quadratic programming.
A \emph{linear program} is a linear constrained optimization problem. Such a problem can be stated in several
different forms, one of which is
\begin{align*}
\text{minimize}\qquad &\c\trp \x \\
\text{subject to}\qquad &G\x \preceq \mathbf{h}\\
&A\x = \b.
\end{align*}

The symbol $\preceq$ denotes that the components of $G\x$ are less than the components of $\mathbf{h}$. In other words, if $\x\preceq\y$, then $x_i < y_i$ for all $x_i\in\x$ and $y_i\in\y$. 

Define vector $\mathbf{s} \succeq \0$ such that the constraint $G\x + \mathbf{s} = \mathbf{h}$. 
This vector is known as a \emph{slack variable}. 
Since $\mathbf{s} \succeq \0$, the constraint
$G\x + \mathbf{s} = \mathbf{h}$ is equivalent to $G\x \preceq \mathbf{h}$.

With a slack variable, a new form of the linear program is found:
\begin{align*}
\text{minimize}\qquad &\c\trp \x \\
\text{subject to}\qquad &G\x + \mathbf{s} = \mathbf{h}\\
&A\x = \b \\
&\mathbf{s} \succeq \0.
\end{align*}

This is the formulation used by CVXOPT.
In this formulation, we require that the matrix $A$ has full row rank,
and that the block matrix $[G \quad A]\trp $ has full column rank.

% \preceq \succeq

% Students have not yet learned about the dual problem. May be included in a later lab.
\begin{comment}
The corresponding \emph{dual program} for the above linear program has the form
\begin{align*}
\text{maximize}\qquad &-h\trp z - b\trp y \\
\text{subject to}\qquad &G\trp z + A\trp y + c = 0\\
 &z \geq 0.
\end{align*}
CVXOPT provides functions to solve both the original (\emph{primal}) linear program and its dual program.
\end{comment}

Consider the following example:
\begin{align*}
\text{minimize}\qquad &-4x_1-5x_2 \\
\text{subject to}\qquad &x_1+2x_2 \leq 3 \\
	        &2x_1+x_2 = 3 \\
		&x_1, x_2 \geq 0
\end{align*}
Recall that all inequalities must be less than or equal to, such that $G\x\preceq \mathbf{h}$.
Because the final two constraints are $x_1, x_2 \geq 0$, they need to be adjusted to be $\leq$ constraints.
This is easily done by multiplying by $-1$, resulting in the constraints $-x_1, -x_2 \leq 0$.
If we define
\[
G = \begin{bmatrix}
  1 & 2\\
  -1 & 0\\
  0 & -1
\end{bmatrix} \text{, } \qquad
\mathbf{h} = \begin{bmatrix}
  3\\
  0\\
  0
\end{bmatrix} \text{, } \qquad
A = \begin{bmatrix}
2 & 1
\end{bmatrix} \text{, } \quad \text{and } \qquad
\mathbf{b} = \begin{bmatrix}
3
\end{bmatrix}
\]
then we can express the constraints compactly as
\[
\begin{matrix}
G\x \preceq \mathbf{h},\\
A\x = \mathbf{b},
\end{matrix}  \qquad \text{where} \qquad
\x = \begin{bmatrix}
  x_1\\
  x_2
\end{bmatrix}.
\]
By adding a slack variable $\mathbf{s}$, we can write our constraints as
\[
G\x + \mathbf{s} = \mathbf{h},
\]
which matches the form discussed above.
% In the case of this particular example, we ignore the extra constraint
%\[
%A\x = \b,
%\]
%since we were given no equality constraints.

To solve the problem using CVXOPT, initialize the arrays $\c$, $G$, $\mathbf{h}$, $A$, and $\mathbf{b}$ and pass them to the appropriate function.
CVXOPT uses its own data type for an array or matrix. 
While similar to the NumPy array, it does have a few differences, especially when it comes to initialization.
Below, we initialize CVXOPT matrices for $\mathbf{c}$, $G$, $\mathbf{h}$, $A$, and $\mathbf{b}$.
We then use the CVXOPT function for linear programming \li{solvers.lp()}, which accepts $\c$, $G$, $\mathbf{h}$, $A$, and $\b$ as arguments.

\begin{lstlisting}
>>> from cvxopt import matrix, solvers

>>> c = matrix([-4., -5.])
>>> G = matrix([[1., -1., 0.],[2., 0., -1.]])
>>> h = matrix([ 3., 0., 0.])
>>> A = matrix([[2.],[1.]])
>>> b = matrix([3.])

>>> sol = solvers.lp(c, G, h, A, b)
     pcost       dcost       gap    pres   dres   k/t
 0: -8.5714e+00 -1.4143e+01  4e+00  0e+00  3e-01  1e+00
 1: -8.9385e+00 -9.2036e+00  2e-01  3e-16  1e-02  3e-02
 2: -8.9994e+00 -9.0021e+00  2e-03  3e-16  1e-04  3e-04
 3: -9.0000e+00 -9.0000e+00  2e-05  1e-16  1e-06  3e-06
 4: -9.0000e+00 -9.0000e+00  2e-07  1e-16  1e-08  3e-08
Optimal solution found.
>>> print(sol['x'])
[ 1.00e+00]
[ 1.00e+00]
>>> print(sol['primal objective'])
-8.999999939019435
>>> print(type(sol['x']))
<<<class 'cvxopt.base.matrix'>>>
\end{lstlisting}

\begin{warn}
CVXOPT matrices only accept floats. 
Other data types will raise a \li{TypeError}.

Additionally, CVXOPT matrices are initialized column-wise rather than row-wise (as in the case of NumPy).
Alternatively, we can initialize the arrays first in NumPy (a process with which you should be familiar),
and then simply convert them to the CVXOPT matrix data type.
\begin{lstlisting}
>>> import numpy as np

>>> c = np.array([-4., -5.])
>>> G = np.array([[1., 2.],[-1., 0.],[0., -1]])
>>> h = np.array([3., 0., 0.])
>>> A = np.array([[2., 1.]])
>>> b = np.array([3.])

# Convert the arrays to the CVXOPT matrix type.
>>> c = matrix(c)
>>> G = matrix(G)
>>> h = matrix(h)
>>> A = matrix(A)
>>> b = matrix(b)
\end{lstlisting}
In this lab, we will initialize non-trivial matrices first as NumPy arrays for consistency.

%Finally, be sure the entries in the matrices are floats!
\end{warn}

%Having initialized the necessary objects, we are now ready to solve the problem.

\begin{info}
Although it is often helpful to see the progress of each iteration of the algorithm, you may suppress this output by first running,
\begin{lstlisting}
solvers.options['show_progress'] = False
\end{lstlisting}
\end{info}

The function \li{solvers.lp()} returns a dictionary containing useful information.
For now, we will only focus on the value of $\x$ and the primal objective value (i.e. the minimum value achieved by the objective function).

\begin{warn}
Note that the minimizer \li{x} returned by the \li{solvers.lp()} function is a \li{cvxopt.base.matrix} object.
\li{np.ravel()} is a NumPy function that takes an object and returns its values as a flattened NumPy array.
Use \li{np.ravel()} to return all minimizers in this lab as flattened NumPy arrays.
\end{warn}

\begin{problem}
Solve the following convex optimization problem:
\begin{align*}
\text{minimize}\qquad &2x_1+x_2+3x_3 \\
\text{subject to}\qquad &x_1+2x_2 \geq 3 \\
	        &2x_1+10x_2+3x_3 \geq 10 \\
		&x_1 \geq 0 \\
		&x_2 \geq 0 \\
		&x_3 \geq 0
\end{align*}
Return the minimizer $\x$ and the primal objective value.
\\(Hint: make the necessary adjustments so that all inequality constraints are $\leq$ rather than $\geq$).
\end{problem}

\subsection*{$l_1$ Norm}
The $l_1$ norm is defined 
\[||\x||_1=\sum_{i=1}^n |x_i|.\]
A $l_1$ minimization problem is minimizing a vector's $l_1$ norm, while fitting certain constraints. It can be written in the following form:
\begin{align*}
\text{minimize}\qquad &\|\x\|_1\\
\text{subject to} \qquad &A\x = \b.
\end{align*}

This problem can be converted into a linear program by introducing an additional vector $\u$ of length $n$.
Define $\u$ such that $|x_i|\leq |u_i|$. 
Thus, $-u_i-x_i\leq 0$ and $-u_i+x_i\leq 0$.
These two inequalities can be added to the linear system as constraints.
Additionally, this means that $||\x||_1\leq ||\u||_1$.
So minimizing $||\u||_1$ subject to the given constraints will in turn minimize $||\x||_1$.
This can be written as follows:
\begin{align*}
\text{minimize}\qquad
&\begin{bmatrix}
\mathbf{1}\trp & \0\trp
\end{bmatrix}
\begin{bmatrix}
\u \\
\x
\end{bmatrix}\\
\text{subject to}\qquad
&\begin{bmatrix}
-I & I\\
-I & -I
\end{bmatrix}
\begin{bmatrix}
\u \\
\x
\end{bmatrix}
\preceq
\begin{bmatrix}
0\\
0
\end{bmatrix},\\
&\begin{bmatrix}
\0 & A
\end{bmatrix}
\begin{bmatrix}
\u \\
\x
\end{bmatrix}
=
\b.
\end{align*}
Solving this gives values for the optimal $\u$ and the optimal $\x$, but we only care about the optimal $\x$.

\begin{problem}
Write a function called \li{l1Min()} that accepts a matrix $A$ and vector $\mathbf{b}$ as NumPy arrays and solves the $l_1$ minimization problem.
Return the minimizer $\x$ and the primal objective value.
Remember to first discard the unnecessary $u$ values from the minimizer.

To test your function consider the matrix $A$ and vector $\mathbf{b}$ below.
\[
A = \begin{bmatrix}
1 & 2 & 1 & 1\\
0 & 3 & -2 & -1
\end{bmatrix} \qquad
\mathbf{b} = \begin{bmatrix}
7 \\
4
\end{bmatrix}
\]
The linear system $A\x = \b$ has infinitely many solutions.
Use \li{l1Min()} to verify that the solution which minimizes $||\mathbf{x}||_1$ is approximately $\x = [1.41, 2.40, 2.40, 0.79]^T$ and the minimum objective value is approximately 7.
\label{prob:l1}
\end{problem}

\section*{The Transportation Problem}

Consider the following transportation problem:
A piano company needs to transport thirteen pianos from their three  supply centers (denoted by 1, 2, 3) to two demand centers (4, 5).
Transporting a piano from a supply center to a demand center incurs a cost, listed in Table \ref{tab:cost}.
The company wants to minimize shipping costs for the pianos while meeting the demand.
%How many pianos should each supply center send to each demand center?

\begin{table}[H]
\centering
\begin{tabular}{|c|c|}
Supply Center & Number of pianos available\\
\hline
1 & 7\\
2 & 2\\
3 & 4\\
\end{tabular}

\caption{Number of pianos available at each supply center}
\label{tab:supply}
\end{table}

\begin{table}[H]
\centering
\begin{tabular}{|c|c|}
Demand Center & Number of pianos needed\\
\hline
4 & 5\\
5 & 8\\
\end{tabular}

\caption{Number of pianos needed at each demand center}
\label{tab:demand}
\end{table}

\begin{table}[H]
\centering
\begin{tabular}{|c|c|c|c|}
Supply Center & Demand Center & Cost of transportation & Number of pianos\\
\hline
1 & 4 & 4 & $p_1$\\
1 & 5 & 7 & $p_2$\\
2 & 4 & 6 & $p_3$\\
2 & 5 & 8 & $p_4$\\
3 & 4 & 8 & $p_5$\\
3 & 5 & 9 & $p_6$\\
\end{tabular}
\caption{Cost of transporting one piano from a supply center to a demand center}
\label{tab:cost}
\end{table}

A system of constraints is defined for the variables $p_1,p_2,p_3,p_4,p_5,$ and $p_6$,
First, there cannot be a negative number of pianos so the variables must be nonnegative.
Next, the Tables \ref{tab:supply} and \ref{tab:demand} define the following three supply constraints and two demand constraints:
\begin{align*}
p_1 + p_2  &= 7\\
p_3 + p_4  &= 2\\
p_5 + p_6  &= 4\\
p_1 + p_3 + p_5 &= 5\\
p_2 + p_4 + p_6 &= 8
\end{align*}

The objective function is the number of pianos shipped from each location multiplied by the respective cost (found in Table \ref{tab:cost}):
\[
4p_1 + 7p_2 + 6p_3 + 8p_4 + 8p_5 + 9p_6.
\]

\begin{info}
Since our answers must be integers, in general this problem turns out to be an NP-hard problem.
There is a whole field devoted to dealing with integer constraints, called \emph{integer linear programming}, which is beyond the scope of this lab.
Fortunately, we can treat this particular problem as a standard linear program and still obtain integer solutions.
\end{info}

Recall the variables are nonnegative, so $p_1,p_2,p_3,p_4,p_5,p_6\geq 0$.
Thus, $G$ and $\mathbf{h}$ constrain the variables to be non-negative.
Because CVXOPT uses the format $G\x \preceq \mathbf{h}$, we see that this inequality must be multiplied by $-1$. 
So, $G$ must be a $6 \times 6$ identity matrix multiplied by $-1$, and

$\mathbf{h}$ is a column vector of zeros.
Since the supply and demand constraints are equality constraints, they are $A$ and $\b$.
Initialize these arrays and solve the linear program by entering the code below.
\begin{lstlisting}
>>> c = matrix(np.array([4., 7., 6., 8., 8., 9.]))
>>> G = matrix(-1*np.eye(6))
>>> h = matrix(np.zeros(6))
>>> A = matrix(np.array([[1.,1.,0.,0.,0.,0.],
                         [0.,0.,1.,1.,0.,0.],
                         [0.,0.,0.,0.,1.,1.],
                         [1.,0.,1.,0.,1.,0.],
                         [0.,1.,0.,1.,0.,1.]]))
>>> b = matrix(np.array([7., 2., 4., 5., 8.]))
>>> sol = solvers.lp(c, G, h, A, b)
     pcost       dcost       gap    pres   dres   k/t
 0:  8.9500e+01  8.9500e+01  2e+01  2e-16  2e-01  1e+00
 1:  8.7023e+01  8.7044e+01  3e+00  1e-15  3e-02  2e-01
Terminated (singular KKT matrix).
>>> print(sol['x'])
[ 4.31e+00]
[ 2.69e+00]
[ 3.56e-01]
[ 1.64e+00]
[ 3.34e-01]
[ 3.67e+00]
>>> print(sol['primal objective'])
87.023
\end{lstlisting}
Notice that some problems occurred. First, CVXOPT alerted us to the fact that the algorithm terminated prematurely (due to a singular matrix).
Second, the minimizer and solution obtained do not consist of integer entries.

So what went wrong? Recall that the matrix $A$ is required to have full row rank, but we can easily see that the rows of $A$
are linearly dependent. We rectify this by converting the last row of the equality constraints into two \emph{inequality} constraints, so that
the remaining equality constraints define a new matrix $A$ with linearly independent rows.

This is done as follows:

 Suppose we have the equality constraint
\[
x_1 + 2x_2 - 3x_3 = 4.
\]
This is equivalent to the pair of inequality
constraints
\begin{align*}
x_1 + 2x_2 - 3x_3 &\leq 4, \\
x_1 + 2x_2 - 3x_3 &\geq 4.
\end{align*}
The linear program requires only $\leq$ constraints, so we obtain the pair
of constraints
\begin{align*}
x_1 + 2x_2 - 3x_3 &\leq 4, \\
-x_1 - 2x_2 + 3x_3 &\leq -4.
\end{align*}

Apply this process to the last equality constraint of the transportation problem.
Then define a new matrix $G$ with several additional rows (to account for the new inequality
constraints), a new vector $\mathbf{h}$ with more entries, a smaller matrix $A$, and a smaller vector $\b$.
\begin{problem}
Solve the transportation problem by converting the last equality constraint into an inequality constraint.
Return the minimizer $\x$ and the primal objective value.
\end{problem}

\begin{comment}
\section*{Example}

Why are all of the terms in $G$ and $\mathbf{h}$ non-positive?

\begin{lstlisting}
>>> from cvxopt import matrix, solvers
>>> G = matrix([ [-1., 0., 0., -1., 0.,  -1., 0., 0., 0., 0., 0.],
             [-1., 0., 0., 0., -1.,  0., -1., 0., 0., 0., 0.],
             [0., -1., 0., -1., 0.,  0., 0., -1., 0., 0., 0.],
             [0., -1., 0., 0., -1.,  0., 0., 0., -1., 0., 0.],
             [0., 0., -1., -1., 0.,  0., 0., 0., 0., -1., 0.],
             [0., 0., -1., 0., -1.,  0., 0., 0., 0., 0., -1.] ])

>>> h = matrix([-7., -2., -4., -5., -8.,  0., 0., 0., 0., 0., 0.,])
>>> c = matrix([4., 7., 6., 8., 8., 9])
>>> sol = solvers.lp(c,G,h)
>>> print sol['x']
>>> print sol['primal objective']
\end{lstlisting}

Another method is to use an integer linear program.
Cvxopt is configured to work with  Gnu, which does have an integer linear program.
It will work with either of the methods above.

\textbf{Example}

glpk.ilp returns a tuple.
The first entry describes the optimality of the result, while the second gives the $x$ values.

\begin{lstlisting}
>>> from cvxopt import matrix, solvers, glpk
>>> G = matrix([ [-1., 0., 0., -1., 0.,  -1., 0., 0., 0., 0., 0.],
             [-1., 0., 0., 0., -1.,  0., -1., 0., 0., 0., 0.],
             [0., -1., 0., -1., 0.,  0., 0., -1., 0., 0., 0.],
             [0., -1., 0., 0., -1.,  0., 0., 0., -1., 0., 0.],
             [0., 0., -1., -1., 0.,  0., 0., 0., 0., -1., 0.],
             [0., 0., -1., 0., -1.,  0., 0., 0., 0., 0., -1.] ])

>>> h = matrix([-7., -2., -4., -5., -8.,  0., 0., 0., 0., 0., 0.,])
>>> o = matrix([4., 7., 6., 8., 8., 9])
>>> sol = glpk.ilp(o,G,h)
>>> print sol[1]
\end{lstlisting}

or
\begin{lstlisting}
>>> from cvxopt import matrix, solvers, glpk
>>> G = matrix([ [-1., 0., 0., 0., 0., 0.],
             [0., -1., 0., 0., 0., 0.],
             [0., 0., -1., 0., 0., 0.],
             [0., 0., 0., -1., 0., 0.],
             [0., 0., 0., 0., -1., 0.],
             [0., 0., 0., 0., 0., -1.] ])

>>> h = matrix([ 0., 0., 0., 0., 0., 0.,])
>>> o = matrix([4., 7., 6., 8., 8., 9])
>>> A = matrix([ [1., 0., 0., 1., 0.],
             [1., 0., 0., 0., 1.],
             [0., 1., 0., 1., 0.],
             [0., 1., 0., 0., 1.],
             [0., 0., 1., 1., 0.],
             [0., 0., 1., 0., 1.] ])
>>> b = matrix([7., 2., 4., 5., 8])
>>> sol = glpk.ilp(o,G,h,A,b)
>>> print sol[1]
\end{lstlisting}

\textbf{Problem 2}
Choose one of these methods and compare the optimal values for the integer linear program to the result you received above.

\textbf{Problem 3}
Create the dual problem for the linear program and solve.
Compare your answer to the dual value cvxopt returned.
\end{comment}

\section*{Quadratic Programming}

Quadratic programming is similar to linear programming, but the objective function is quadratic rather than linear.
The constraints, if there are any, are still of the same form.
Thus, $G, \mathbf{h}, A$, and $\b$ are optional.
The formulation that we will use is
\begin{align*}
\text{minimize}\qquad &\frac{1}{2}\x\trp Q\x + \mathbf{r}\trp \x \\
\text{subject to}\qquad &G\x \preceq \mathbf{h}\\
 &A\x = \b,
\end{align*}
where $Q$ is a positive semidefinite symmetric matrix.
In this formulation, we require again that $A$ has full row rank, and that the block matrix
$[Q \quad G \quad A]\trp $ has full column rank.

As an example, consider the quadratic function
\[
f(x_1,x_2) = 2x_1^2 +2x_1x_2 + x_2^2 +x_1 -x_2.
\]
There are no constraints, so we only need to initialize the matrix $Q$ and the vector $\mathbf{r}$.
To find these, we first rewrite our function to match the formulation given above.
If we let
\[
Q = \begin{bmatrix}
  a & b\\
  b & c
\end{bmatrix}, \qquad
\mathbf{r} = \begin{bmatrix}
  d\\
  e
\end{bmatrix},
\qquad \text{and} \qquad
\x = \begin{bmatrix}
  x_1\\
  x_2
\end{bmatrix},
\]
then
\begin{align*}
\frac{1}{2}\x\trp Q\x + \mathbf{r}\trp \x &=
\frac{1}{2}
\begin{bmatrix}
  x_1\\
  x_2
\end{bmatrix}\trp
\begin{bmatrix}
  a & b\\
  b & c
\end{bmatrix}
\begin{bmatrix}
  x_1\\
  x_2
\end{bmatrix} +
\begin{bmatrix}
  d\\
  e
\end{bmatrix}\trp
\begin{bmatrix}
  x_1\\
  x_2
\end{bmatrix} \\
&= \frac{1}{2}ax_1^2 + bx_1x_2 + \frac{1}{2}cx_2^2 + dx_1 + ex_2
\end{align*}
Thus, we see that the proper values to initialize our matrix $Q$ and vector $\mathbf{r}$ are:
\begin{align*}
a &= 4  &d = 1 \\
b &= 2  &e = -1 \\
c &= 2
\end{align*}
Now that we have the matrix $Q$ and vector $\mathbf{r}$, we are ready to use the CVXOPT function for quadratic programming \li{solvers.qp()}.
\begin{lstlisting}
>>> Q = matrix(np.array([[4., 2.], [2., 2.]]))
>>> r = matrix([1., -1.])
>>> sol=solvers.qp(Q, r)
>>> print(sol['x'])
[-1.00e+00]
[ 1.50e+00]
>>> print sol['primal objective']
-1.25
\end{lstlisting}

\begin{problem}
Find the minimizer and minimum of
\begin{equation*}
g(x_1,x_2,x_3) = \frac{3}{2}x_1^2 +2x_1x_2 + x_1x_3+ 2x_2^2 +2x_2x_3+\frac{3}{2}x_3^2+3x_1 + x_3
\end{equation*}
\\(Hint: Write the function $g$ to match the formulation given above before coding.)
\begin{comment}
\begin{equation}
f(x) = \frac{1}{2}x\trp Qx - x\trp p
\end{equation}
where

\begin{center}
$Q =
\begin{bmatrix}
3 & 2 & 1\\
2 & 4 & 2\\
1 & 2 & 3\\
\end{bmatrix}
$
and $p =
\begin{bmatrix}
3\\
0\\
1\\
\end{bmatrix}
$
\end{center}
\end{comment}

\end{problem}


\begin{problem}
The $l_2$ minimization problem is to
\begin{align*}
\text{minimize}\qquad &\|\x\|_2\\
\text{subject to} \qquad &A\x = \b.
\end{align*}

This problem is equivalent to a quadratic program, since $\|\x\|_2 = \x\trp \x$.
Write a function that accepts a matrix $A$ and vector $\b$ and solves the $l_2$ minimization problem.
Return the minimizer $\x$ and the primal objective value.

To test your function, use the matrix $A$ and vector $\b$ from Problem \ref{prob:l1}. 
The minimizer is approximately $\x=[1.55, 2.36, 2.36, 0.73]^T$ and the minimum primal objective value is approximately 14.09.
\end{problem}

\section*{Allocation Models}
Allocation models lead to simple linear programs. An allocation model seeks to allocate a valuable resource among competing needs. Consider the following example taken from ``Optimization in Operations Research" by Ronald L. Rardin. %%pg 132

The U.S. Forest service has used an allocation model to deal with the task of managing national forests.
The model begins by dividing the land into a set of analysis areas. Several land management policies (also
called prescriptions) are then proposed and evaluated for each area.
An \emph{allocation} is how much land (in acreage) in each unique analysis area will be assigned to each of the possible prescriptions.
We seek to find the best possible allocation, subject to forest-wide restrictions on land use.

The file \li{ForestData.npy} contains data for a fictional national forest (you can also find the data
in Table \ref{tab:forest}). There are 7 areas of analysis and 3 prescriptions for each of them.

\begin{align*}
&\text{Column 1: $i$, area of analysis} \\
&\text{Column 2: $s_i$, size of the analysis area (in thousands of acres)} \\
&\text{Column 3: $j$, prescription number} \\
&\text{Column 4: $p_{i,j}$, net present value (NPV) per acre in area $i$ under prescription $j$} \\
&\text{Column 5: $t_{i,j}$, protected timber yield per acre in area $i$ under prescription $j$} \\
&\text{Column 6: $g_{i,j}$, protected animal grazing capability per acre for area $i$ under prescription $j$} \\
&\text{Column 7: $w_{i,j}$, wilderness index rating (0 to 100) for area $i$ under prescription $j$}
\end{align*}

\begin{table}[H]
\centering
    \begin{tabular}{c c c c c c c}
&&&Forest Data&&& \\
\hline
Analysis & Acres &Prescrip-&NPV&Timber&Grazing&Wilderness \\
Area&(1000)'s &tion&(per acre) &(per acre)&(per acre)& Index\\
$i$ &$s_i$&$j$& $p_{i,j}$ & $t_{i,j}$&$g_{i,j}$&$w_{i,j}$ \\\hline
1&	75	&1	&503	&310	&0.01&	40\\
&&		2&	140&	50&	0.04	&80\\
&&		3&	203&	0&	0&	95\\ \hline
2&	90&	1	&675&	198&	0.03&	55\\
&&		2&	100&	46&	0.06&	60\\
&&		3&	45&	0&	0&	65\\ \hline
3&	140&	1	&630&	210	&0.04&	45\\
&&		2&	105&	57&	0.07&	55\\
&&		3&	40	&0&	0&	60\\ \hline
4	&60&	1&	330&	112&	0.01&	30\\
&&		2	&40&	30&	0.02&	35\\
&&		3&	295&	0&	0	&90\\ \hline
5	&212&	1	&105	&40	&0.05&	60\\
&&		2	&460&	32	&0.08&	60\\
&& 3	&120&0&	0	&70\\ \hline
6	&98	&1	&490	&105	&0.02	&35\\
&&		2&	55	&25	&0.03	&50\\
&&		3	&180	&0	&0	&75\\ \hline
7&	113&	1	&705	&213&	0.02	&40\\
&&		2&	60	&40	&0.04&	45\\
&&		3	&400	&0	&0	&95\\
\hline
    \end{tabular}
\caption{}
\label{tab:forest}
\end{table}
Let $x_{i,j}$ be the amount of land in area $i$ allocated to prescription $j$.
Under this notation, an allocation is a one-dimensional vector consisting of the $x_{i,j}$'s. 
For this particular
example, there are 7 acres, with 3 prescriptions each.
So the allocation vector is a one-dimensional vector with 21 entries.
Our goal is to find the allocation vector that maximizes net present value, while producing at least 40 million
board-feet of timber, at least 5 thousand units of grazing capability, and keeping the average wilderness index at least 70.
The allocation vector is also constrained to be nonnegative, and all of the land must be allocated precisely.

Since acres are in thousands, divide the constraints of timber and animal grazing by 1000 in the problem setup, and compensate for this after obtaining a solution.

The problem can be written as follows:
\begin{align*}
\text{maximize } &\sum\limits_{i=1}^7 \sum\limits_{j=1}^3 p_{i,j}x_{i,j} \\
\text{subject to } &\sum\limits_{j=1}^3 x_{i,j} = s_i  \text{ for } i=1,..,7 \\
	        &\sum\limits_{i=1}^7 \sum\limits_{j=1}^3 t_{i,j}x_{i,j} \geq 40,000 \\
		&\sum\limits_{i=1}^7 \sum\limits_{j=1}^3 g_{i,j}x_{i,j} \geq 5 \\
		&\frac{1}{788} \sum\limits_{i=1}^7 \sum\limits_{j=1}^3 w_{i,j}x_{i,j} \geq 70 \\
		&x_{i,j} \geq 0 \text{ for } i=1,...,7  \text{ and } j=1,2,3
\end{align*}

\begin{problem}
Solve the allocation problem above.
Return the minimizing allocation vector of $x_{i,j}$'s and the maximum total net present value.
Remember to consider the following:
\begin{enumerate}
\item The allocation vector should be a (21,1) NumPy array.
\item Recall that the constraints of timber and animal grazing were divided by 1000.
To compensate, the maximum total net value will be equal to the primal objective of the appropriately minimized linear function multiplied by -1000.
\end{enumerate}
\end{problem}

You can learn more about CVXOPT at
\url{http://cvxopt.org/index.html}.
