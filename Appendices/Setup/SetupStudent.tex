\ifwindows
\lab{Getting Started On Windows}{Getting Started On Windows}
\else
\lab{Getting Started On Mac/Linux}{Getting Started On Mac/Linux}
\fi

% Make tildes look nicer in the code boxes
\lstset{
    literate={~} {{\raisebox{.6ex}{\texttildelow}}}{1}
}

Welcome to ACME!
This guide will help you get set up for
\ifbootcamp
ACME.
\else
your summer workflow.
\fi


\ifwindows
\section*{Installing WSL}

Windows Subsystem for Linux (WSL) is a compatibility layer for running Linux natively on Windows 10 and 11. 
We'll go through the steps to install Ubuntu on Windows using WSL, which you will use to run all your code in ACME.
You may have already developed your own workflow on Windows that you're familiar with, but we highly recommend that you use WSL, as not doing so will most certainly give you extensive grief when trying to install different items for many ACME labs.

\noindent Don't worry if you have zero Linux experience. 
Even with very little or no Linux experience, you will soon become comfortable enough to help your peers and coworkers with it in the future.
\begin{enumerate}
\item \emph{Enabling WSL:}
\begin{enumerate}
    \item Open the Start menu and search for ``Turn Windows features on or off''. 
    \item Click on it.
    \item In the ``Windows Features'' window, scroll down until you see ``Windows Subsystem for Linux'' in the list. 
    \item Check the box next to it to enable the feature.
    \item You should also enable ``Virtual Machine Platform''.
    \item Click ``OK'' to save your changes. 
\end{enumerate}

\item \emph{Installing Ubuntu:}
Now that WSL is enabled, you can install Ubuntu on your computer.
\begin{enumerate}
\item Open the Microsoft Store app on your computer.
\item Search for ``Ubuntu'' in the search bar and select ``Ubuntu'' from the search results.
\item Click the ``Get'' button to download and install Ubuntu on your computer.
\item Wait for the installation to complete.
\end{enumerate}

\item \emph{Launching Ubuntu:}
Now that Ubuntu is installed, we can launch it and start using it.
\begin{enumerate}
\item Press the Windows key and type ``Ubuntu'' in the search bar. 
\item Select ``Ubuntu'' from the search results to launch it.
\item Wait for Ubuntu to start up and create a new user account with a username and password when prompted.
\end{enumerate}
Make sure not to forget your password!

\item \emph{Installing Linux Updates:}
\begin{enumerate}
\item In the Ubuntu terminal, run the command \li{sudo apt update} to update the package list.
\item Run the command \li{sudo apt upgrade} to install any necessary updates.
\end{enumerate}
% \item If you plan on accessing the Windows filesystem from within Ubuntu, you'll need to configure the Windows username and password by running the command "sudo nano /etc/wsl.conf" and adding the following lines to the file:

% \begin{lstlisting}[language=bash, backgroundcolor=\color{black}, basicstyle=\color{white}]
% [user]
% default=username
% \end{lstlisting}

% Replace "username" with your Windows username.

% \item Save and exit the file by pressing "Ctrl+X", then "Y", and finally "Enter".
\end{enumerate}


\section*{Finding Your Files in WSL}

WSL has its own independent home and file system apart from Windows, but you can still access your files in Windows from WSL by navigating back through the \li{mnt} directory.
For example, to access your \li{c} drive (your files) you can use \li{cd /mnt/c} to navigate to your standard head directory.
So if you want to access your \li{Desktop}, you can find it at \li{/mnt/c/Users/<username>/Desktop}.
% In order to find your Documents, you can use the command \li{ls /mnt/c/Users/<username>/OneDrive/Documents}.
This may feel counterintuitive to how Windows normally works, but this one difference is a small price to pay to make your life much easier.

Fortunately, you can actually change the default ``boot'' directory that Ubuntu starts in.
The process for changing your boot directory is very simple, and it is recommended that you do so to make your life easier.
\begin{enumerate}
    \item Open your Ubuntu terminal.
    \item Type \li{cd ~} to go to your home directory.
    \item Type \li{sudo nano .profile} to open your profile file.
    \item Add the path of the desired default directory to the end of the file.
    For example, if you wanted to change the default ``boot'' directory to be your base user directory, you would add the following line of code to the end of the file:
\begin{lstlisting}[language=bash]
cd /mnt/c/Users/<username>
\end{lstlisting}
    where \li{<username>} is the name of your user.
    \item Save and exit the file by pressing \li{Ctrl+X}, then \li{Y}, and finally \li{Enter}.
    \item Close your terminal and reopen it.
\end{enumerate}
Ubuntu should now immediately start in your user directory.
\fi


\ifbootcamp
\section*{Git}

Git is a version control system that helps you manage changes to your code over time. 
This section will serve as a guide for the later steps, so do not worry about learning everything right now. 
It will come with time.
Git allows you to keep track of different versions of your code, collaborate with others, and revert changes if necessary.

To install Git on WSL or another Linux system, run the following command in 
\ifwindows
Ubuntu:
\else
your terminal:
\fi
\begin{lstlisting}[language=bash]
sudo apt install git
\end{lstlisting}
\ifwindows

\else
On Mac, you can just type \li{git} into your terminal and it will prompt you to install it.
If you have an M1 Mac, you should also double check that you have OS version 12 or later.
\fi

\section*{Application in ACME}

In ACME, Git is not only used to save and organize your work, but also to turn in assignments.
The most current version of your code will be \emph{pulled} by the instructor at a predesignated time and graded.
If you always \emph{push} your code when you're done working on it, you will never miss turning in an assignment.
Explanations of these Git terms are provided later in this document.
\fi


%%%%%%%%%%%%%%%%%%%%%%%%%%%%%%%%%%%%%%%%%%%%%%%%%%%%%%%%%%%%%%
\ifbyu  % if byu-specific
\section*{Downloading Course Materials}

The lab manuals can be found at 
\url{https://acme.byu.edu/}.
You can also download the zip folders that contain the lab materials from this page.
\ifbootcamp
However, you won't need to download them since the method we will use to set up your GitHub repository will not require them.
There is a more involved way to set everything up (found in the Additional Materials section), which involves downloading the zip files and pushing your materials up to a GitHub repository, but the following procedure is much simpler.
\fi

\ifbootcamp
\section*{Online Setup}
\begin{enumerate}
    \item \emph{Sign up for GitHub}.
    \label{step:sign-up}
    If you already have a GitHub account, sign into your account and proceed to the next step.
    Otherwise, create a GitHub account at \url{https://github.com/}.
    This is where you will store your files online for grading.
  
%   You may next be asked to make a workspace. 
%   If you are, using your name for the workspace will make everything much easier for the grader.
%   If you do not receive this prompt, it will likely default to your name.

    \item \emph{Set up your Git credentials.}
    You will need to set up your your Git \li{user.name} and \li{user.email}.
    \li{user.name} should be your actual name, and \li{user.email} must be the email associated with your GitHub account.
    To do this, open 
    \ifwindows
    Ubuntu,
    \else 
    your terminal,
    \fi
    and run the following code:
\begin{lstlisting}
$ git config --global user.name "your name"
$ git config --global user.email "your email"
\end{lstlisting}    

    \item \emph{Create an ssh key}.
    This step only needs to be done once on each computer you want to use to access your repository.
    If you have multiple repositories on the same computer, you do \emph{not} need to repeat this step for each one.
    To create an ssh key, enter the following command in
    \ifwindows
    Ubuntu:
    \else
    the terminal:
    \fi
\begin{lstlisting}
ssh-keygen -t ecdsa -b 256
\end{lstlisting}
    Press the \li{Enter} or \li{Return} key to accept the default file location.
    It will then prompt you to enter a Password, but you can press \li{Enter} or \li{Return} again to skip this step if you don't want a Password.
    The key will then be created.
    The file for the key will be placed in the \li{/home/<username>/.ssh} (or \li{\~/.ssh}) directory.
  
    Now that the key is created, you need to add it to your GitHub account.
    From GitHub, click on your profile icon in the upper right corner, and click on \textbf{Settings} towards the bottom of the menu.
    Now click on \textbf{SSH and GPG keys} on the left, and click the green \textbf{New SSH key} button.
    Make a Title for this key (it doesn't matter what you put, but you may wish to specify which machine this key will be used for).
    Make sure the \textbf{Key type} is set to \texttt{Authentication Key}.

    Now, using the file explorer, navigate to the \li{.ssh} folder, and open the \emph{public key} file.
    This file should be called \li{id_ecdsa.pub}; do \emph{NOT} use \li{id_ecdsa} (without the \li{.pub} extension).
    Copy the contents of this file and paste it into the \texttt{Key} field on GitHub. 
  
    If you're having trouble navigating to the \li{.ssh} folder, you can try to print the contents of the folder by running this code in your terminal:
\begin{lstlisting}
cat ~/.ssh/id_ecdsa.pub 
\end{lstlisting}
    Once you've pasted the contents of \li{id_ecdsa.pub} into GitHub, press the green \textbf{Add SSH key} button.
    Then, in your terminal run 
\begin{lstlisting}
ssh-add ~/.ssh/id_ecdsa 
\end{lstlisting}
    If you get an error that says ``Could not open a connection to your authentication agent,'' then run 
\begin{lstlisting}
eval $(ssh-agent)
ssh-add ~/.ssh/id_ecdsa 
\end{lstlisting}

    To verify that this worked, when you make your first commit the new ssh key should be added to the file \li{\~/.ssh/known_hosts}.
  
%   For more options and some troubleshooting information, refer to \href{https://docs.github.com/en/authentication/connecting-to-github-with-ssh/adding-a-new-ssh-key-to-your-github-account}{this link}.
  
    \item \emph{Clone GitHub Classroom Repository}.
    In this step, you will set up a GitHub repository with all of your lab materials.
    The following steps will guide you through the easiest way to accomplish this, in which you will simply clone a template repository from GitHub Classroom.

    Repeat this procedure for both Volume 1 and Volume 2.
    \begin{itemize}
        \item On \emph{Learning Suite}, you will find a link to a GitHub Classroom.
        Open the link in a new tab.
        \item Click the green \textbf{Accept this assignment} button.
        \item After a few seconds, refresh the page.
        The page should now say ``You're ready to go!''
        Beneath the line that reads ``Your assignment repository has been created:'' there should be a GitHub link.
        Click the link.
        \item You will be directed to your personal GitHub repository, which you must now clone onto your computer.
        To clone the repository, click on the green \textbf{Code} button.
        Make sure the \textbf{SSH} option is selected, then copy the link below it.
        \item Now, in
        \ifwindows
        Ubuntu,
        \else
        your terminal,
        \fi
        navigate to a good location on your computer where you will want all your lab materials (i.e. \li{Documents}, \li{Desktop}, etc.), then type \li{git clone} followed by a space and then the link you copied from GitHub.
        It should look something like 
\begin{lstlisting}
git clone git@github.com:classroom_name/repo_name.git
\end{lstlisting}
        Then press \li{Enter} or \li{Return}.
        An identical clone of the GitHub repository should now download onto your computer.
    \end{itemize}

    \item \emph{Download data files}.
    \label{step:download-data}
    Many labs have accompanying data files.
    To download these files, navigate into each of your course directories and run the \li{download\_data.sh} bash script, which downloads the files and places them in the correct lab folders for you.
    You can also find individual data files through \href{https://github.com/Foundations-of-Applied-Mathematics/Student-Materials/wiki/Lab-Index}{\texttt{Student-Materials/wiki/Lab-Index}}.
\begin{lstlisting}
# Navigate to your folder and run the script
$ <b<cd>b> /path/to/folder
$ bash download_data.sh
\end{lstlisting}
    Repeat this process for both Volume 1 and Volume 2.

\end{enumerate}
\fi

%%%%%%%%%%%%%%%%%%%%%%%%%%%%%%%%%%%%%%%%%%%%%%%%%%%%%%%%%%%%%%
\else  % if non-byu specific
\section*{Downloading Course Materials}
The lab manuals can be found at \url{https://foundations-of-applied-mathematics.github.io/}.
You can also download the zip folders that contain the lab materials from this page.
You should then unzip these in a folder you are familiar with, such as your \li{Documents} or \li{Desktop} folder.
\begin{warn}
Make absolutely sure your unzipped lab folders are not nested!
When you open each lab folder, you should NOT have to open another folder to access all the lab materials!
If your folder is nested, move the inner folder out so that there's only ``one layer'' to access your lab materials.
\end{warn}

\ifbootcamp
\section*{Online Setup}
\begin{enumerate}
    \item \emph{Sign up for GitHub}.
    \label{step:sign-up}
    If you already have a GitHub account, sign into your account and proceed to the next step.
    Otherwise, create a GitHub account at \url{https://github.com/}.
    This is where you will store your files online for grading.
  
%   You may next be asked to make a workspace. 
%   If you are, using your name for the workspace will make everything much easier for the grader.
%   If you do not receive this prompt, it will likely default to your name.

    \item \emph{Create an ssh key}.
    This step only needs to be done once on each computer you want to use to access your repository.
    If you have multiple repositories on the same computer, you do \emph{not} need to repeat this step for each one.
    To create an ssh key, enter the following command in
    \ifwindows
    Ubuntu:
    \else
    the terminal:
    \fi
\begin{lstlisting}
ssh-keygen -t ecdsa -b 256
\end{lstlisting}
    Press the \li{Enter} or \li{Return} key to accept the default file location.
    It will then prompt you to enter a Password, but you can press \li{Enter} or \li{Return} again to skip this step if you don't want a Password.
    The key will then be created.
    The file for the key will be placed in the \li{/home/<username>/.ssh} (or \li{\~/.ssh}) directory.
  
    Now that the key is created, you need to add it to your GitHub account.
    From GitHub, click on your profile icon in the upper right corner, and click on \textbf{Settings} towards the bottom of the menu.
    Now click on \textbf{SSH and GPG keys} on the left, and click the green \textbf{New SSH key} button.
    Make a Title for this key (it doesn't matter what you put, but you may wish to specify which machine this key will be used for).
    Make sure the \textbf{Key type} is set to \texttt{Authentication Key}.

    Now, using the file explorer, navigate to the \li{.ssh} folder, and open the \emph{public key} file.
    This file should be called \li{id_ecdsa.pub}; do \emph{NOT} use \li{id_ecdsa} (without the \li{.pub} extension).
    Copy the contents of this file and paste it into the \texttt{Key} field on GitHub. 

    If you're having trouble navigating to the \li{.ssh} folder, you can try to print the contents of the folder by running this code in your terminal:
\begin{lstlisting}
cat ~/.ssh/id_ecdsa.pub 
\end{lstlisting}
    Once you've pasted the contents of \li{id_ecdsa.pub} into GitHub, press the green \textbf{Add SSH key} button.
    Then, in your terminal run 
\begin{lstlisting}
ssh-add ~/.ssh/id_ecdsa 
\end{lstlisting}
    If you get an error that says ``Could not open a connection to your authentication agent,'' then run 
\begin{lstlisting}
eval $(ssh-agent)
ssh-add ~/.ssh/id_ecdsa 
\end{lstlisting}

    To verify that this worked, when you make your first commit the new ssh key should be added to the file \li{\~/.ssh/known_hosts}.
  
%   For more options and some troubleshooting information, refer to \href{https://docs.github.com/en/authentication/connecting-to-github-with-ssh/adding-a-new-ssh-key-to-your-github-account}{this link}.
  
    \item \emph{Make a new repository}.
    You will need to create a new repository for each lab folder you downloaded.
    Note: you \emph{must} follow these instructions exactly. 
    If you do not complete each step exactly as specified below, delete the repository and start over.
    \begin{itemize}
        \item On your GitHub Home page, click the green \textbf{Create repository} button to the upper left of the screen; or, if you have already used GitHub before and already have repositories set up, click on the green \textbf{New} button next to \textbf{Top Repositories}.
        \item First you will need to provide a Repository name; this can be anything you want, but please make it relevant to the lab folder it will represent; you may also provide a repository description if you like.
        \item Mark the repository as \textbf{Private}.
        \item Leave the box next to \textbf{Add a README file} unchecked (if you accidentally include a \texttt{README}, delete the repository and start over).
        % \item For \textbf{Default branch name}, enter \texttt{main}. 
        \item Under \textbf{Add .gitignore}, select \textbf{None} for \texttt{.gitignore template} (if you accidentally include a \texttt{.gitignore}, delete the repository and start over).
        \item Under \textbf{Choose a license}, select \textbf{None} for \texttt{License}.
        % \item Under \textbf{Advanced settings}, enter a short description for your repository, select \textbf{No forks} under forking, and select \textbf{Python} as the language.
        \item Finally, click the green \textbf{Create repository} button.
    \end{itemize}
    Repeat this process for each lab folder you downloaded.
  
    \item \emph{Give the instructor access to your repository (First Day of Class)}.
    If you are doing this before the first day of class, skip to the next step.
    In your newly created repository, click the \textbf{Settings} tab along the top of the page and click on \textbf{Collaborators} on the left.
    Then click on the green \textbf{Add people} button.
    Type your instructor's GitHub username, select them, then click the big green \textbf{Add <username> to this repository} button.

    \item \emph{Connect your folder to the new repository}.
    \label{step:connect-folder}
    In 
    \ifwindows
    Ubuntu
    \else
    your terminal
    \fi 
    enter the following commands.
\begin{lstlisting}
# Navigate to your folder.
$ <b<cd>b> /path/to/folder  # cd means 'change directory'.

# Make sure you are in the right place.
$ <b<pwd>b>                 # pwd means 'print working directory'.
/path/to/folder
$ <b<ls>b> *.md             # ls means 'list files'.
README.md             # This means README.md is in the working directory.

# Connect this folder to the online repository.
$ git init
$ git remote add origin git@github.com:<username>/<repo>.git
# Make sure the link has this form. If it starts with https, select the ssh option on GitHub instead.

# Record your credentials. user.name should be your name, and user.email must be the email associated with your GitHub account.
$ git config --global user.name "your name"
$ git config --global user.email "your email"

# Add the contents of this folder to Git and update the repository.
$ git add --<<all>>
$ git commit -m "initial commit"
$ git branch -M main    # Update branch name.
$ git push origin main
\end{lstlisting}
    
    For example, if your GitHub username is \li{greek314}, the repository is called \texttt{acmev1}, and the folder is called \texttt{Student-Materials/} and is located in \li{Desktop}, you would enter the following commands:
\begin{lstlisting}
# Navigate to the folder.
$ <b<cd>b> ~/Desktop/Student-Materials

# Make sure this is the right place.
$ <b<pwd>b>
/Users/Archimedes/Desktop/Student-Materials
$ <b<ls>b> *.md
README.md

$ git init
$ git remote add origin git@github.com:greek314/acmev1.git

$ git config --global user.name "archimedes"
$ git config --global user.email "greek314@example.com"

$ git add --<<all>>
$ git commit -m "initial commit"
$ git branch -M main
$ git push origin main
\end{lstlisting}
    
    At this point you should be able to see the same files on your GitHub repository that are also found on your machine in your course directory.
    If you enter the repository URL incorrectly in the \li{git remote add origin} step, you can reset it with the following line:
\begin{lstlisting}
$ git remote <<set>>-url origin git@github.com:<username>/<repo>.git
\end{lstlisting}

    \begin{info}
    You may get an error like the following when you run \li{git push}:
\begin{lstlisting}
...
<<fatal: Authentication failed for 'https://github.com/<username>/<repo>.git/'>>
\end{lstlisting}
    If this error occurs, your repository URL is in the wrong format; most likely, you used the \li{https} format instead of the \li{ssh} format shown above.
    You can use the \li{<<git remote set-url origin>>} command to fix this issue as well.
    \end{info}
    
    \item \emph{Download data files}.
    \label{step:download-data}
    Many labs have accompanying data files.
    To download these files, navigate into each of your course directories and run the \li{download\_data.sh} bash script, which downloads the files and places them in the correct lab folders for you.
    You can also find individual data files through \href{https://github.com/Foundations-of-Applied-Mathematics/Student-Materials/wiki/Lab-Index}{\texttt{Student-Materials/wiki/Lab-Index}}.
    
\begin{lstlisting}
# Navigate to your folder and run the script
$ <b<cd>b> /path/to/folder
$ bash download_data.sh
\end{lstlisting}
    Repeat this process for each of your lab folders.

\end{enumerate}
\fi
\fi
%%%%%%%%%%%%%%%%%%%%%%%%%%%%%%%%%%%%%%%%%%%%%%%%%%%%%%%%%%%%%


\section*{Using VSCode}
VSCode is the recommended code editor in ACME, though it is not required.
We recommended it because it's free and
\ifwindows
open source, and it can use WSL very easily.
\else
open source.
\fi
You can download it at \url{https://code.visualstudio.com/}.
Once you have it installed, you can search for it on your machine under: \li{Visual Studio Code}.
You can also open it from the terminal by typing \li{code}.
\ifwindows
\begin{enumerate}
    \item \emph{Installing the Remote WSL Extension:}
    The Remote WSL extension allows you to use VSCode to edit files in WSL.
    You can install it by opening VSCode and clicking on the \li{Extensions} tab on the left (little building blocks).
    Search for \li{WSL}.
    Then click install.

    You will need to restart VSCode for the extension to take effect.

    \item \emph{Opening a Folder in WSL:}
    In order to open a folder in WSL, click on the bottom left corner of VSCode (blue box) to Open a Remote Window.
    Then click \li{Connect to WSL}.
    Now click on the \li{Explorer} icon in the upper left (two pages), and click \li{Open Folder}.
    If you see 
    \begin{lstlisting}[language=bash]
    /home/<username>
    \end{lstlisting}
    in the searchbar, you are accessing the WSL home directory, not the one where your files are.
    Type in the complete file path to
    \ifbootcamp
    one of your
    \ifbyu
    course directories,
    \else
    lab folders,
    \fi 
    such as \li{/mnt/c/Users/<username>/Desktop/Volume1}.
    \else 
    your course directory, such as \li{/mnt/c/Users/<username>/Desktop/SummerLabs}.
    \fi
    Make sure to navigate into the folder that contains all the lab materials, not just \li{Desktop} or another higher directory.
    Click \li{ok} to save this path.
    \ifbootcamp
    \ifbyu
    Repeat these steps for the other Volume.
    \else 
    Repeat these steps for the other lab folders.
    \fi
    \fi
\end{enumerate}
\fi


\section*{Installing Python}
\ifwindows
\else
    Before installing Python on Mac, you will need to install Homebrew (brew).
    To check if brew is already installed, run
\begin{lstlisting}
which brew
\end{lstlisting}
    If the answer is 
\begin{lstlisting}
<</usr/local/bin/brew>>
\end{lstlisting}
    then you can skip to the update / upgrade step, but if the answer is
\begin{lstlisting}[language=bash]
brew not found
\end{lstlisting}
    then you must first install brew by running the command found at \href{brew.sh}{brew.sh}, which will probably look like
\begin{lstlisting}[language=bash]
/bin/bash -c "$(curl -fsSL https://raw.githubusercontent.com/Homebrew/install/HEAD/install.sh)"
\end{lstlisting}
    You can update brew by running
\begin{lstlisting}[language=bash]
brew update
brew upgrade
\end{lstlisting}
\fi

To install Python, navigate into
\ifbootcamp
one of your course directories (e.g. \li{Volume1}) 
\else 
your course directory (e.g. \li{SummerLabs})
\fi
and run 
\begin{lstlisting}[language=bash]
bash install_python.sh
\end{lstlisting}
Now restart your terminal.
On Windows or Linux, you will need to run the following commands in WSL or Standard Linux Terminal:
\begin{lstlisting}[language=bash]
sudo apt-get install graphviz
sudo apt-get install libopenmpi-dev
sudo apt-get install ffmpeg
sudo apt-get install g++
sudo apt-get install python3-tk
sudo apt-get install openjdk-8-jdk
\end{lstlisting}
Assuming no errors, you should now run
\begin{lstlisting}[language=bash]
bash install_dependencies.sh
\end{lstlisting}
You have now installed Python and all necessary dependencies you will need in ACME!

\ifbootcamp
\ifbyu
\section*{Setup on a Lab Machine}
Most ACME students prefer to do their work on their own machines.
However, throughout the semester something may happen where you are unable to install something, or your computer might break, and it will be easier to use a different computer.
For this reason, it is recommended that you set up your repository on a lab machine as well.
The software on the lab machine will always work, and if it somehow doesn't, you will have dedicated IT help with the lab computers.

The simplest way to set up a lab machine is to log into one of them in person, but you can also access one remotely through the \emph{Secure Shell Protocol} (ssh).
If you are unfamiliar with the ssh process, don't worry; you will learn this skill and many others shortly in the ACME program.
For now, we will explain some ssh basics (or if you have physical access to a lab machine, you may skip this paragraph).
To ssh onto a lab machine, open 
\ifwindows
Ubuntu
\else
your terminal
\fi
and type:
\begin{lstlisting}[language=bash]
ssh <netID>@acme<number_between_01_and_30>.byu.edu
\end{lstlisting}
You will then be prompted for your password; this will be your BYU password.
If this doesn't work, try another number (it will not matter which number you end up using).
When you ssh onto a lab machine, your terminal gains control of the lab computer.
You can verify this by checking that the machine name before the \$ is now different than before and says ``acme''.

Now that you have access to a lab machine, you will need to repeat the step \emph{Create an ssh key} from the \emph{Online Setup} section.
This will involve uploading another ssh key to GitHub specific to this lab machine.
Next, you will need to clone your repository onto the lab machine.
To do this, first navigate into the \emph{myacmeshare} folder by running 
\begin{lstlisting}
cd myacmeshare
\end{lstlisting}
in your terminal.
Then, go to GitHub and navigate to the repository you wish to clone.
Find and click the green \textbf{Code} button.
Make sure that \textbf{SSH} is underlined and copy the link; then run 
\begin{lstlisting}
git clone <link>
\end{lstlisting} 
in your terminal.
This will clone your repository onto the lab machine.

You will also need to repeat the step \emph{Download data files} from the \emph{Online Setup} section.
Once you have repeated this process for both volumes, you are ready to go.
You can now access your repository on the lab machine.
Now, if anything goes wrong, you can either ssh onto the lab machine or use it in person, and you will be able to access and work on your ACME labs without a problem.

You can also use VSCode to ssh onto the lab machine in a similar way to connecting to WSL.
To do this, install the Remote-SSH extension using the \li{Extensions} tab on the left (little building blocks) of the window in VSCode.
Then, you can add a Remote Window by clicking on the blue \li{Open a Remote Window} button in the bottom left corner, selecting \emph{Connect To Host}, and typing in \li{<netID>@acme<number_between_01_and_30>.byu.edu}.
\fi

\subsection*{If All Else Fails\dots}
If, while working on a lab, you can't get something to work on 
\ifbyu
both your own computer and the lab machine,
\else 
your computer,
\fi
your last resort is Google Colab.
It should be reserved as a last resort, because Google Colab is formatted in a notebook style, which is fundamentally different than a \li{.py} file.
However, if you are ever stuck, keep in mind that it's still an option.
See the Using Google Colab guide in the Appendix for more information on using Colab for ACME labs.
\fi


\ifbootcamp
\section*{Additional Git Help}
Here are some key concepts and terminology you'll need to understand and use Git effectively.
You may need to refer back to this multiple times.
\begin{itemize}
\item \emph{Repositories.}
A Git repository is a collection of files and folders that Git is tracking. 
By creating a repository, you are simply telling the software that it should back up certain files that you tell it to.
When you create a repository, Git creates a hidden directory called \li{.git} inside your project folder that tells it to track those files.

\item \emph{Adding Files.}
Git will only track the files that you specifically add.
You can add files individually, or you can add all the files in a folder at once.
To add a file, you'll typically use the command:
\begin{lstlisting}[language=bash]
git add <filename>
\end{lstlisting}

\item \emph{Commits.}
A commit is a snapshot of your code at a specific point in time. 
When you make changes to your code, you can create a new commit to record those changes. 
Each commit has a unique identifier, which allows you to reference it later if you need to revert your code to a previous state.
If you are collaborating with others, it gives you an ``undo'' button specific to each user, allowing you to be more organized.
It is good practice to commit every time you leave your computer to keep track of your work.

To create a commit, you'll typically use the command:
\begin{lstlisting}[language=bash]
git commit -m "Commit message"
\end{lstlisting}

\item \emph{Branches.}
A branch is a separate line of development that diverges from the main line of development. 
By creating a new branch, you can work on a feature or bug fix without affecting the main codebase. 
Once you're done with your changes, you can merge the branch back into the main codebase.

You probably won't branch your code off the main branch in the ACME courses, but this term is still useful to know.

\item \emph{Cloning.}
Cloning a repository is simply downloading a codebase.
Just like downloading a zip folder, all the files are present, but the original author does not have access to your files unless you push them back.
% It may seem more difficult than a zip file at first, but the ability to directly and immediately backup what you are doing to the cloud makes the extra effort worth it.

To clone a repository, you'll typically use the command:
\begin{lstlisting}[language=bash]
git clone <url>
\end{lstlisting}

\item \emph{Pushing and Pulling.}
Among the most frequent operations you'll use with Git are pushing and pulling.

When you \emph{push} changes to a remote repository, you're sending your commits to a central server where others can access them.
This will be where the grader accesses your code in order to give you a grade. 
To push your changes, you'll typically use the command:

\begin{lstlisting}[language=bash]
git push <remote> <branch>
\end{lstlisting}

Here, ``remote'' refers to the remote repository where you want to push your changes (e.g., ``origin''), and ``branch'' refers to the name of the branch you're pushing (e.g., ``main'').
Good practice is to push every few hours of working, and absolutely every time you're done with a lab.
Pushing saves your work on a secure server, meaning if something catastrophic happens to your computer, none of your work will be lost and you can continue where you left off on a different machine.
Do not assume you're different and that computer issues will never happen to you.
Doing a lab from scratch twice in a week is guaranteed to take more time than just once!

\begin{figure}[H]
    \centering
    \begin{tikzpicture}
        % Define styles
        \tikzstyle{Box}=[rectangle,draw=black!35,font=\sffamily\footnotesize,align=center,minimum height=.8cm,minimum width=2.75cm]
        \tikzstyle{Command}=[rectangle,draw=none,fill=none,font=\sffamily\footnotesize,align=center]
        % Rectangles
        \node[Box,fill=black!10!white,text=black!80!white] at (0,1.5) (B1) {Online Repository};
        \node[Box,fill=black!10!white,text=black!80!white] at (0,-1.5) (B2) {Computer};
        % Words (commands)
        \node[Command] at (-2.5,0) (C1) {\lif{git push origin main}};
        \node[Command] at (2.5,0) (C2) {\lif{git pull origin main}};
        % Lines
        \foreach \o/\i/\a/\b in {0/90/B1/C2,180/270/B2/C1} \draw[help lines,line width=.75pt,shorten >=-.1cm] (\a) to [out=\o,in=\i] (\b);
        % Arrows
        \foreach \o/\i/\a/\b in {270/0/C2/B2,90/180/C1/B1} \draw[->,>=stealth',help lines,line width=.75pt,shorten <=-.1cm] (\a) to [out=\o,in=\i] (\b);
    \end{tikzpicture}
    \caption*{Exchanging Git commits between the repository and a local clone.}
\end{figure}

When you \emph{pull} changes from a remote repository, you're downloading changes that others have made and incorporating them into your local codebase. 
This will be used to pull your grading feedback back from the grader.
To pull changes, you'll typically use the command:
\begin{lstlisting}[language=bash]
git pull <remote> <branch>
\end{lstlisting}
Here, ``remote'' and ``branch'' have the same meanings as in the \li{git push} command.
Good practice is to pull just before any time you push. 

\item \emph{Origin and Main.}
``Origin'' and ``main'' are two terms you'll often see when working with Git.

``Origin'' refers to the default remote repository where your code is stored. 
When you clone a repository, Git sets up a remote called ``origin'' that points to the URL of the original repository. 
You can push/pull changes to/from ``origin'' to collaborate with others.

``Main'' is the name of the default branch in a Git repository. 
This is the branch where the main line of development occurs, and it's typically the branch you'll be working on most of the time. 
However, depending on the repository, the default branch may be named something else (e.g., ``master'').
\end{itemize}
\begin{table}[H]
    \begin{tabular}{l|l}
        Command & Explanation \\ \hline
        \li{git status} & Display the staging area and untracked changes. \\
        \li{git pull origin main} & Pull changes from the online repository. \\
        \li{git push origin main} & Push changes to the online repository. \\
        \li{git add <filename(s)>} & Add a file or files to the staging area. \\
        \li{git add -u} & Add all modified, tracked files to the staging area. \\
        \li{git commit -m "<message>"} & Save the changes in the staging area with a given message. \\
        \li{git checkout -- <filename>} & Revert changes to an unstaged file since the last commit. \\
        \li{git reset HEAD -- <filename>} & Remove a file from the staging area. \\
        \li{git diff <filename>} & See the changes to an unstaged file since the last commit. \\
        \li{git diff --cached <filename>} & See the changes to a staged file since the last commit. \\
        \li{git config --local <option>} & Record your credentials (\li{user.name}, \li{user.email}, etc.). \\
    \end{tabular}
    \caption*{Common Git commands.}
\end{table}

Now that you've learned about some of Git's features, it's time to implement them in your first ACME Lab, \emph{Intro to GitHub}!
\fi

\newpage

\ifbyu
\ifbootcamp
\section*{Additional Material} % ==============================================

\section*{Manual GitHub Setup}

\section*{Downloading Course Materials}
The lab manuals can be found at 
\ifbyu
\url{https://acme.byu.edu/}.
\else
\url{https://foundations-of-applied-mathematics.github.io/}
\fi
\ifbootcamp
If you are a Junior, you should also download the Volume 1 and 2 zip folders. 
\else
If you are a Junior, you should also download the Junior Summer materials.
\fi
You should then unzip 
\ifbootcamp
these
\else
this
\fi
in a folder you are familiar with, such as your \li{Documents} or \li{Desktop} folder.
\begin{warn}
Make absolutely sure your unzipped lab folders are not nested!
When you open each lab folder, you should NOT have to open another folder to access all the lab materials!
If your folder is nested, move the inner folder out so that there's only ``one layer'' to access your lab materials.
\end{warn}

\section*{Online Setup With Lab Folders}
\begin{enumerate}
    \item \emph{Sign up for GitHub}.
    \label{step:sign-up}
    If you already have a GitHub account, sign into your account and proceed to the next step.
    Otherwise, create a GitHub account at \url{https://github.com/}.
    This is where you will store your files online for grading.
  
%   You may next be asked to make a workspace. 
%   If you are, using your name for the workspace will make everything much easier for the grader.
%   If you do not receive this prompt, it will likely default to your name.

    \item \emph{Create an ssh key}.
    This step only needs to be done once on each computer you want to use to access your repository.
    If you have multiple repositories on the same computer, you do \emph{not} need to repeat this step for each one.
    To create an ssh key, enter the following command in
    \ifwindows
    Ubuntu:
    \else
    the terminal:
    \fi
\begin{lstlisting}
ssh-keygen -t ecdsa -b 256
\end{lstlisting}
    Press the \li{Enter} or \li{Return} key to accept the default file location.
    It will then prompt you to enter a Password, but you can press \li{Enter} or \li{Return} again to skip this step if you don't want a Password.
    The key will then be created.
    The file for the key will be placed in the \li{/home/<username>/.ssh} (or \li{\~/.ssh}) directory.
  
    Now that the key is created, you need to add it to your GitHub account.
    From GitHub, click on your profile icon in the upper right corner, and click on \textbf{Settings} towards the bottom of the menu.
    Now click on \textbf{SSH and GPG keys} on the left, and click the green \textbf{New SSH key} button.
    Make a Title for this key (it doesn't matter what you put, but you may wish to specify which machine this key will be used for).
    Make sure the \textbf{Key type} is set to \texttt{Authentication Key}.

    Now, using the file explorer, navigate to the \li{.ssh} folder, and open the \emph{public key} file.
    This file should be called \li{id_ecdsa.pub}; do \emph{NOT} use \li{id_ecdsa} (without the \li{.pub} extension).
    Copy the contents of this file and paste it into the \texttt{Key} field on GitHub. 

    If you're having trouble navigating to the \li{.ssh} folder, you can try to print the contents of the folder by running this code in your terminal:
\begin{lstlisting}
cat ~/.ssh/id_ecdsa.pub 
\end{lstlisting}
    Once you've pasted the contents of \li{id_ecdsa.pub} into GitHub, press the green \textbf{Add SSH key} button.
    Then, in your terminal run 
\begin{lstlisting}
ssh-add ~/.ssh/id_ecdsa 
\end{lstlisting}
    If you get an error that says ``Could not open a connection to your authentication agent,'' then run 
\begin{lstlisting}
eval $(ssh-agent)
ssh-add ~/.ssh/id_ecdsa 
\end{lstlisting}

    To verify that this worked, when you make your first commit the new ssh key should be added to the file \li{\~/.ssh/known_hosts}.
  
%   For more options and some troubleshooting information, refer to \href{https://docs.github.com/en/authentication/connecting-to-github-with-ssh/adding-a-new-ssh-key-to-your-github-account}{this link}.
  
    \item \emph{Make a new repository}.
    You will need to create a new repository for both Volume 1 and Volume 2.
    Note: you \emph{must} follow these instructions exactly. 
    If you do not complete each step exactly as specified below, delete the repository and start over.
    \begin{itemize}
        \item On your GitHub Home page, click the green \textbf{Create repository} button to the upper left of the screen; or, if you have already used GitHub before and already have repositories set up, click on the green \textbf{New} button next to \textbf{Top Repositories}.
        \item First you will need to provide a Repository name; this can be anything you want, but please make it relevant to the ACME Volume it will represent; you may also provide a repository description if you like.
        \item Mark the repository as \textbf{Private}.
        \item Leave the box next to \textbf{Add a README file} unchecked (if you accidentally include a \texttt{README}, delete the repository and start over).
        % \item For \textbf{Default branch name}, enter \texttt{main}. 
        \item Under \textbf{Add .gitignore}, select \textbf{None} for \texttt{.gitignore template} (if you accidentally include a \texttt{.gitignore}, delete the repository and start over).
        \item Under \textbf{Choose a license}, select \textbf{None} for \texttt{License}.
        % \item Under \textbf{Advanced settings}, enter a short description for your repository, select \textbf{No forks} under forking, and select \textbf{Python} as the language.
        \item Finally, click the green \textbf{Create repository} button.
    \end{itemize}
    Repeat this process for both Volume 1 and Volume 2.
  
    \item \emph{Give the instructor access to your repository (First Day of Class)}.
    If you are doing this before the first day of class, skip to the next step.
    In your newly created repository, click the \textbf{Settings} tab along the top of the page and click on \textbf{Collaborators} on the left.
    Then click on the green \textbf{Add people} button.
    Type your instructor's GitHub username, select them, then click the big green \textbf{Add <username> to this repository} button.

    \item \emph{Connect your folder to the new repository}.
    \label{step:connect-folder}
    In 
    \ifwindows
    Ubuntu
    \else
    your terminal
    \fi 
    enter the following commands.
\begin{lstlisting}
# Navigate to your folder.
$ <b<cd>b> /path/to/folder  # cd means 'change directory'.

# Make sure you are in the right place.
$ <b<pwd>b>                 # pwd means 'print working directory'.
/path/to/folder
$ <b<ls>b> *.md             # ls means 'list files'.
README.md             # This means README.md is in the working directory.

# Connect this folder to the online repository.
$ git init
$ git remote add origin git@github.com:<username>/<repo>.git
# Make sure the link has this form. If it starts with https, select the ssh option on GitHub instead.

# Record your credentials. user.name should be your name, and user.email must be the email associated with your GitHub account.
$ git config --global user.name "your name"
$ git config --global user.email "your email"

# Add the contents of this folder to Git and update the repository.
$ git add --<<all>>
$ git commit -m "initial commit"
$ git branch -M main    # Update branch name.
$ git push origin main
\end{lstlisting}
    
    For example, if your GitHub username is \li{greek314}, the repository is called \texttt{acmev1}, and the folder is called \texttt{Student-Materials/} and is located in \li{Desktop}, you would enter the following commands:
\begin{lstlisting}
# Navigate to the folder.
$ <b<cd>b> ~/Desktop/Student-Materials

# Make sure this is the right place.
$ <b<pwd>b>
/Users/Archimedes/Desktop/Student-Materials
$ <b<ls>b> *.md
README.md

$ git init
$ git remote add origin git@github.com:greek314/acmev1.git

$ git config --global user.name "archimedes"
$ git config --global user.email "greek314@example.com"

$ git add --<<all>>
$ git commit -m "initial commit"
$ git branch -M main
$ git push origin main
\end{lstlisting}
    
    At this point you should be able to see the same files on your GitHub repository that are also found on your machine in your course directory.
    If you enter the repository URL incorrectly in the \li{git remote add origin} step, you can reset it with the following line:
\begin{lstlisting}
$ git remote <<set>>-url origin git@github.com:<username>/<repo>.git
\end{lstlisting}

    \begin{info}
    You may get an error like the following when you run \li{git push}:
\begin{lstlisting}
...
<<fatal: Authentication failed for 'https://github.com/<username>/<repo>.git/'>>
\end{lstlisting}
    If this error occurs, your repository URL is in the wrong format; most likely, you used the \li{https} format instead of the \li{ssh} format shown above.
    You can use the \li{<<git remote set-url origin>>} command to fix this issue as well.
    \end{info}
    
    \item \emph{Download data files}.
    \label{step:download-data}
    Many labs have accompanying data files.
    To download these files, navigate into each of your course directories and run the \li{download\_data.sh} bash script, which downloads the files and places them in the correct lab folders for you.
    You can also find individual data files through \href{https://github.com/Foundations-of-Applied-Mathematics/Student-Materials/wiki/Lab-Index}{\texttt{Student-Materials/wiki/Lab-Index}}.
    
\begin{lstlisting}
# Navigate to your folder and run the script
$ <b<cd>b> /path/to/folder
$ bash download_data.sh
\end{lstlisting}
    Repeat this process for both Volume 1 and Volume 2.

\end{enumerate}
\fi
\fi
