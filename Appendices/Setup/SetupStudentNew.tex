\ifwindows
\lab{Getting Started On Windows}{Getting Started On Windows}
\else
\lab{Getting Started On Mac/Linux}{Getting Started On Mac/Linux}
\fi
% \usepackage{listings}
% \usepackage{xcolor}
% \usepackage{hyperref}
%TODO Mac version 12
Welcome to ACME!

\noindent This "Getting Started" guide will help you set up
\ifbootcamp
the labs.
\else
your workflow for summer.
\fi

% The labs in this curriculum aim to introduce computational and mathematical concepts, walk through implementations of those concepts in Python, and use industrial-grade code to solve relevant problems.
% Lab assignments are usually about 5--10 pages long and include code examples (yellow boxes), important notes (green boxes), warnings about common errors (red boxes), and about 3--7 exercises (blue boxes).
\ifwindows
\section*{Installing WSL}

Windows Subsystem for Linux (WSL) is a compatibility layer for running Linux natively on Windows 10/11. 
We'll go through the steps to install Ubuntu on Windows using WSL which will be used to run all relevant programs.
Many people have their own workflows they are familiar with, and while we do not discourage these, we warn you that you may have extensive difficulty installing different items for different labs.

\noindent Do not worry if you have zero Linux experience. Even with little or no Linux experience, you will soon become comfortable enough with it to help your peers and coworkers in the future.
\begin{enumerate}
\item \emph{Enabling WSL:}
\begin{enumerate}
    \item Open the Start menu and search for "Turn Windows features on or off". 
    \item Click on it.
    \item In the "Windows Features" window, scroll down until you see "Windows Subsystem for Linux" in the list. 
    \item Check the box next to it to enable the feature.
    \item Click "OK" to save your changes. 
\end{enumerate}

\item \emph{Installing Ubuntu:}
Now that WSL is enabled, you can install Ubuntu on your computer.
\begin{enumerate}
\item Open the Microsoft Store app on your computer.
\item Search for "Ubuntu" in the search bar and select "Ubuntu" from the search results.
\item Click the "Get" button to download and install Ubuntu on your computer.
\item Wait for the installation to complete.
\end{enumerate}

\item \emph{Launching Ubuntu:}
Now that Ubuntu is installed, we can launch it and start using it.
\begin{enumerate}
\item Press the Windows key and type "Ubuntu" in the search bar. 
\item Select "Ubuntu" from the search results to launch it.
\item Wait for Ubuntu to start up and create a new user account with a username and password when prompted.
\end{enumerate}
Make sure not to forget the password.

\item \emph{Installing Linux Updates:}
\begin{enumerate}

\item In the Ubuntu terminal, run the command "sudo apt update" to update the package list.
\item Install any necessary updates by running the command "sudo apt upgrade".
\end{enumerate}
% \item If you plan on accessing the Windows filesystem from within Ubuntu, you'll need to configure the Windows username and password by running the command "sudo nano /etc/wsl.conf" and adding the following lines to the file:

% \begin{lstlisting}[language=bash, backgroundcolor=\color{black}, basicstyle=\color{white}]
% [user]
% default=username
% \end{lstlisting}

% Replace "username" with your Windows username.

% \item Save and exit the file by pressing "Ctrl+X", then "Y", and finally "Enter".
\end{enumerate}
\fi
\ifbootcamp
\section*{Git}

Git is a version control system that helps you manage changes to your code over time. 
This section will serve as a guide for the later steps, so do not worry about learning everything right now. 
It will come with time.
It allows you to keep track of different versions of your code, collaborate with others, and revert changes if necessary.

To install git on Windows and Linux, in 
\ifwindows
Ubuntu
\else
your terminal
\fi
run the command:
\begin{lstlisting}[language=bash]
sudo apt install git
\end{lstlisting}
\ifwindows

\else
On Mac, you can just type \li{git} into your terminal and it will prompt you to install it.
If you have an M1 Mac, you should also double check you have OS version 12 or later.
\fi

\section*{Application in ACME}

In ACME, git is not only used to save and organize your work, but also turn in assignments.
The most current version of your code will be pulled by the instructor at a predesignated time and graded.
If you always push your code when you're done working on it, you will never miss turning in an assignment.
Explanations of these terms are provided later in this document.
\fi
\ifwindows
\section*{Finding Your Files From WSL}
\begin{warn}
To access your c drive (your files) you will need to use \li{cd /mnt/c} to get to the head directory.
\end{warn}
This is counterintuitive to how Windows normally works, but other than this everything will be easier.
Essentially WSL has its own home and file system, but once you switch over to the Windows side you will still have all of your files.
For example, if you want to access your Desktop, you can find it at \li{/mnt/c/Users/<username>/Desktop}.
In order to find your documents, you can use the command \li{ls /mnt/c/Users/<username>/OneDrive/Documents}.
If you see 
\begin{lstlisting}[language=bash]
/home/<username>
\end{lstlisting}
You are accessing the WSL home directory, not the one where your files are.

The process for changing your boot directory is very simple, and it is recommended you do it now.
\begin{enumerate}
    \item Open your Ubuntu terminal.
    \item Type \li{cd ~} to go to your home directory.
    \item Type \li{sudo nano .bashrc} to open your bashrc file.
    \item Add the following line to the end of the file:
    \begin{lstlisting}[language=bash]
    cd /mnt/c/Users/<username>/OneDrive/Documents
    \end{lstlisting}
    \item Save and exit the file by pressing "Ctrl+X", then "Y", and finally "Enter".
    \item Close your terminal and reopen it.
\end{enumerate}
\fi
You should now be in a more familiar directory from the start.
\section*{Downloading Course Materials}
The lab manuals can be found at \url{https://acme.byu.edu/}.
\ifbootcamp
If you are a junior, you should download the volume 1 and 2 zips. 
\else
If you are a junior, you should download the junior summer materials.
\fi
You should then unzip 
\ifbootcamp
these
\else
this
\fi
in a folder you are familiar with, such as your documents folder or desktop folder.

\section*{Using VSCode}

VSCode is the recommended code editor in ACME, though it is not required.
We recommended it because it is free
\ifwindows
, open source, and can use WSL very easily.
\else
and open source.
\fi
You can download it at \url{https://code.visualstudio.com/}.
Once you have it installed, you can open it by typing in the search bar \li{Visual Studio Code}.
You can also open it from the terminal by typing \li{code}.
\ifwindows
\begin{enumerate}
\item \emph{Installing the Remote WSL Extension:}
The Remote WSL extension allows you to use VSCode to edit files in WSL.
You can install it by opening VSCode then click on the Extensions tab on the left (little building blocks).
Search for \li{WSL}.
Then click install.

You will need to restart VSCode for the extension to take effect.

\item \emph{Opening a Folder in WSL:}
In order to open a folder in WSL, click on the bottom left corner of VSCode (blue box) to open a remote window.
Then click \li{Connect to WSL}.
Type in the file path such as \li{/mnt/c/Users/<username>/Desktop/Volume1}.

Make sure to navigate to the folder that contains all of the lab folders, not just your desktop or another higher directory.
    %TODO Explain this better.
Click open.
\ifbootcamp
Repeat these steps for the other volume.
\fi

\end{enumerate}
\fi
\ifbootcamp
\section*{Online Setup}
\begin{enumerate}
  \item \emph{Sign up for Bitbucket}.
  \label{step:sign-up}
  Create a Bitbucket account at \url{https://bitbucket.org}.
  This is where you will store your files online.
  You should use your netid@byu.edu address for this, both to get their free academic benefits, and because it is a university requirement.
  You may next be asked to make a workspace. 
  If you are, using your name for the workspace will make everything much easier for the grader.
  If you do not receive this prompt, it will likely default to your name.
  
  \item \emph{Make a new repository}.
  \begin{itemize}
    \item On the Bitbucket page, click on \textbf{Create} then select \textbf{Repository}.
    \item First you will need to provide a project name and this can be anything you want, though please make it relevant.
    \item Then you will need to provide a name for the repository and mark the repository as \textbf{private}.
    
    \item For \textbf{Include a README?}, select \textbf{No} (if you accidentally include a \texttt{README}, delete the repository and start over).
    \item For \textbf{Default branch name}, enter \texttt{main}. 
    \item For \textbf{Include .gitignore?}, select \textbf{No} (if you accidentally include a \texttt{.gitignore}, delete the repository and start over).
    \item Under \textbf{Advanced settings}, enter a short description for your repository, select \textbf{No forks} under forking, and select \textbf{Python} as the language.
    \item Finally, click the blue \textbf{Create repository} button.
    \end{itemize}
  
  \item \emph{Give the instructor access to your repository (First Day of Class)}.
  If you are doing this before the first day of class, skip to the next step.
  On your newly created Bitbucket repository page (\texttt{https://bitbucket.org/<netid>/<repo>} or similar), go to \textbf{Repository Settings} in the menu to the left and select \textbf{Repository permissions} and then \textbf{Add users or groups}.
  Enter your instructor's Bitbucket username under \textbf{Users} and change the permission from read to write, then click \textbf{Add}.
  
  \item \emph{Create an SSH key}.
  This step needs to be done only once on each computer that you want to be able to use to access your repository.
  If you have multiple repositories on the same computer, you do \emph{not} need to repeat this step for each one.
  To create an SSH key, enter the following command in
  \ifwindows
  Ubuntu
  \else
  the terminal
  \fi
  :
  \begin{lstlisting}
  $ ssh-keygen -t ecdsa -b 256
  \end{lstlisting}
  Press the Enter or Return key to accept the default file location.
  It will then prompt to enter a passphrase; this acts as a password to use the SSH key.
  If you do not want a passphrase, leave it blank and press Enter again.
  The key will then be created.
  The file for the key will be placed in in the \li{/home/<username>/.ssh} directory.
  
  Now that the key is created, you need to add it to your Bitbucket account.
  From Bitbucket, click the settings icon in the top right corner, then \textbf{Personal settings} and then \textbf{SSH keys}.
  Click \textbf{Add key} and enter a label (what it is doesn't matter).
  Now, using the file explorer, navigate to the SSH key you created, and open the \emph{public key} file.
  The file will be called something like \li{id_rsa.pub}; do \emph{NOT} use \li{id_rsa} (without the \li{.pub} extension).
  Copy the contents of this file, paste it into the Key field on Bitbucket, and press Add Key.
  
  For more options and some troubleshooting information, refer to \url{https://support.atlassian.com/bitbucket-cloud/docs/set-up-an-ssh-key/}.
  
  \item \emph{Connect your folder to the new repository}.
  \label{step:connect-folder}
  In 
  \ifwindows
  Ubuntu
  \else
  your terminal
  \fi 
  enter the following commands.
  \begin{lstlisting}
    # Navigate to your folder.
    $ <b<cd>b> /path/to/folder  # cd means 'change directory'.
    
    # Make sure you are in the right place.
    $ <b<pwd>b>                 # pwd means 'print working directory'.
    /path/to/folder
    $ <b<ls>b> *.md             # ls means 'list files'.
    README.md             # This means README.md is in the working directory.
    
    # Connect this folder to the online repository.
    $ git init
    $ git remote add origin git@bitbucket.org:<name>/<repo>.git
    # Make sure the link has this form. If it starts with https, select the SSH option on bitbucket instead.
    
    # Record your credentials.
    $ git config --local user.name "your name"
    $ git config --local user.email "your email"
    
    # Add the contents of this folder to git and update the repository.
    $ git add --<<all>>
    $ git commit -m "initial commit"
    $ git push origin main
    \end{lstlisting}
    
    For example, if your Bitbucket username is \li{greek314}, the repository is called \texttt{acmev1}, and the folder is called \texttt{Student-Materials/} and is on the desktop, enter the following commands.
    
    \begin{lstlisting}
    # Navigate to the folder.
    $ <b<cd>b> ~/Desktop/Student-Materials
    
    # Make sure this is the right place.
    $ <b<pwd>b>
    /Users/Archimedes/Desktop/Student-Materials
    $ <b<ls>b> *.md
    README.md
    
    # Connect this folder to the online repository.
    $ git init
    $ git remote add origin git@bitbucket.org:greek314/acmev1.git
    
    # Record credentials. user.name can be anything, but user.email must be the email associated with your Bitbucket account.
    $ git config --local user.name "archimedes"
    $ git config --local user.email "greek314@example.com"
    
    # Add the contents of this folder to git and update the repository.
    $ git add --<<all>>
    $ git commit -m "initial commit"
    $ git push origin main
    \end{lstlisting}
    
    At this point you should be able to see the files on your repository page from a web browser.
    If you enter the repository URL incorrectly in the \li{git remote add origin} step, you can reset it with the following line:
    \begin{lstlisting}
    $ git remote <<set>>-url origin git@bitbucket.org:<netid>/<repo>.git
    \end{lstlisting}
    \begin{info}
    You may get the an error like the following when you run \li{git push}:
    \begin{lstlisting}
    <<remote: Bitbucket Cloud recently stopped supporting account passwords for Git authentication.>>
    ...
    <<fatal: Authentication failed for 'https://bitbucket.org/<netid>/<repo>.git/'>>
    \end{lstlisting}
    If this error occurs, your repository URL is in the wrong format; most likely, you used the \li{https} version instead of what is shown above (\li{ssh}).
    You can use the \li{<<git remote set-url origin>>} command to fix this issue as well.
    \end{info}
    
    \item \emph{Download data files}.
    \label{step:download-data}
    Many labs have accompanying data files.
    To download these files, navigate to your clone and run the \texttt{download\_data.sh} bash script, which downloads the files and places them in the correct lab folder for you.
    You can also find individual data files through \href{https://github.com/Foundations-of-Applied-Mathematics/Student-Materials/wiki/Lab-Index}{\texttt{Student-Materials/wiki/Lab-Index}}.
    
    \begin{lstlisting}
        # Navigate to your folder and run the script.
        $ <b<cd>b> /path/to/folder
        $ bash download_data.sh
    \end{lstlisting}
\end{enumerate}
\fi
\section*{Installing Python}

To install python on Windows or Linux, you will need to run the following commands in WSL/Standard Linux Terminal.
\begin{lstlisting}[language=bash]
	sudo apt-get install ffmpeg
	sudo apt-get install libopenmpi-dev
	sudo apt-get install g++
	sudo apt-get install python3-tk
	sudo apt-get install openjdk-8-jdk
\end{lstlisting}
\ifwindows
\else
    On mac, you will need to install homebrew.
    To check if brew is already installed, type
    \begin{lstlisting}
    which brew
    \end{lstlisting}.
    If the answer is 
    \begin{lstlisting}
    /usr/local/bin/brew 
    \end{lstlisting}
    then you're good to go update / upgrade, but if the answer is
    \begin{lstlisting}[language=bash]
    brew not found
    \end{lstlisting}
    then you must first install brew.
    You will do this by typing
    \begin{lstlisting}[language=bash]
    /bin/bash -c "$(curl -fsSL https://raw.githubusercontent.com/Homebrew/install/HEAD/install.sh)"
    \end{lstlisting}
    into terminal.
    You can update brew using 
    \begin{lstlisting}[language=bash]
    brew update
    brew upgrade
    \end{lstlisting}.
\fi
Assuming you are in your course directory (i.e. \li{Volume1}), you can now install python with the command
\begin{lstlisting}[language=bash]
    bash install_python.sh
\end{lstlisting}
Assuming no errors, you should now restart your terminal and then run
\begin{lstlisting}[language=bash]
    bash install_dependencies.sh
\end{lstlisting}
You have now installed python and all the necessary dependencies for all of ACME.
% You can now install python with the command
% \begin{lstlisting}[language=bash]
% brew install python@3.10; brew install pip; brew install ipython
% \end{lstlisting}
% \fi
% There are three things there.
% \begin{enumerate}
%     \item Python, which is the python language itself.
%     \item Pip, which is a package manager for python.
%     \item IPython, which is an interactive python shell.
% \end{enumerate}

% If something is not included in the base python installation, you can use pip to install it.
% You will know you need to install something if you encounter a ModuleNotFoundError, which you may encounter soon.
% For example, in order to install numpy, you could write \li{pip3 install numpy} in the terminal.
% If this does not work on your machine, try \li{pip install numpy} or \li{python3 -m pip install numpy}.

% IPython is an interactive python shell.
% You will be taught how to use it soon.

% \noindent Python will now be accessible from the terminal with the command \li{python3}.
% You can also run a python script with \li{python3 <scriptname>}.

% \ifwindows
% Python is now accessable globally, meaning you can run python from any directory as long as you are still typing the command in ubuntu.
% \else
% Python is now accessable globally, meaning you can run python from any directory as long as you are still typing the command in the terminal.
% \fi
\ifbootcamp

\section*{Setting up on a Lab Machine}
Most people prefer to do their work on their own machine.
However, throughout the semester something may happen where you are unable to install something, or your computer will break, and it will be easier to use a different computer.
For this reason, it is a good idea to set up your repository on a lab machine as well.
Software will always work, and if it does not, you will have dedicated IT help with the lab computers.

In order to set up everything on the lab machine, you will first need to ssh onto the machine.
To do this, open 
\ifwindows
Ubuntu
\else
your terminal
\fi
and type:
\begin{lstlisting}[language=bash]
ssh <netID>@acme<number between 01 and 30>.byu.edu
\end{lstlisting}
You will then be prompted for your password.
This will be your BYU password.
If this does not work, try another number. 
It does not matter which number you use.

If you are unfamiliar with this process, do not worry.
These skills will be introduced in detail soon.

When you ssh onto the machine, your terminal will now control the lab computer.
You can make sure of this by the machine name before the \$ being different than before and saying acme.
First, you will need to repeat the step \emph{Create an SSH key} from the earlier section.
Once you have completed that section and upload your ssh key again to bitbucket, you are ready for the last section.

Now, you will need to clone your repository onto the lab machine.
To do this, first navigate to the \emph{myacmeshare} folder by typing \li{cd myacmeshare}.
Then, you will go to bitbucket, navigate to your new repository, and find the clone button in the top right corner.
Copy the link and type \li{git clone <link>} in the terminal.
This will clone your repository onto the lab machine.

You will need to repeat the step \emph{Download data files} from the earlier section.
Once you have repeated this section for both volumes, you are ready to go.
You can now access your repository on the lab machine.

If anything goes wrong, you can either ssh onto the lab machine or use the lab machine in person.

Another good hint is that you can use VSCode to ssh onto the lab machine.
You will do this using the same process as connecting to WSL.
You will need to install the Remote-SSH extension using the small boxes on the left side of the window of VSCode.
Then you will add a remote by clicking on the bottom left corner, selecting \emph{Connect To Host}, and typing in \li{<netID>@acme<number between 01 and 30>.byu.edu}.
\fi
\ifbootcamp
\section{Additional Git Help}
\noindent Here are some key concepts and terminology you'll need to understand to use Git effectively.
\noindent You may need to refer back to this multiple times.
\begin{itemize}
\item \emph{Repositories}
A Git repository is a collection of files and folders that Git is tracking. 
By creating a repository, you are simply telling the software that it should back up certain files that you tell it to.
When you create a repository, Git creates a hidden directory called ".git" inside your project folder that tells it that those files should be tracked.

\item \emph{Adding Files}
Git add is the command you use to tell Git which files you want to track.
You can add files individually, or you can add all of the files in a folder at once.
To add a file, you'll typically use the command:
\begin{lstlisting}[language=bash]
git add <filename>
\end{lstlisting}

\item \emph{Commits}
A commit is a snapshot of your code at a specific point in time. 
When you make changes to your code, you can create a new commit to record those changes. 
Each commit has a unique identifier, which allows you to reference it later if you need to revert your code to a previous state.
If you are collaborating with others, it gives you an "undo" button specific to each user, allowing you to be more organized.
It is good practice to commit every time you leave your computer to keep track of your work.

To create a commit, you'll typically use the command:
\begin{lstlisting}[language=bash]
git commit -m "Commit message"
\end{lstlisting}

\item \emph{Branches}
A branch is a separate line of development that diverges from the main line of development. 
By creating a new branch, you can work on a feature or bug fix without affecting the main codebase. 
Once you're done with your changes, you can merge the branch back into the main codebase.

You probably won't branch your code off the main branch in these courses, but this term is still useful to know.
\item \emph{Cloning}
Cloning a repository is simply downloading a codebase.
Just like downloading a zip folder, all of the files are present, but the original author does not have access to your files unless you push them to them.
% It may seem more difficult than a zip file at first, but the ability to directly and immediatedly backup what you are doing to the cloud makes the extra effort worth it.

To clone a repository, you'll typically use the command:
\begin{lstlisting}[language=bash]
git clone <url>
\end{lstlisting}

\item \emph{Pushing and Pulling}
Pushing and pulling are two operations you'll use frequently when working with Git.

When you push changes to a remote repository, you're sending your commits to a central server where others can access them.
This will be where the grader accesses your code in order to give you a grade. 
To push your changes, you'll typically use the command:

\begin{lstlisting}[language=bash]
git push <remote> <branch>
\end{lstlisting}

Here, "remote" refers to the remote repository where you want to push your changes (e.g., "origin"), and "branch" refers to the name of the branch you're pushing (e.g., "main").
Good practice is to push every few hours of working, and absolutely every time you are done with a lab.
Pushing saves your work on a secure server, meaning if something catastrophic happens to your computer, none of your work will be lost and you can continue where you left off on a different machine.
Do not assume you are different and computer issues will not happen to you.
Doing a lab from scratch twice in a week is guaranteed to take more time than just once.

When you pull changes from a remote repository, you're downloading changes that others have made and incorporating them into your local codebase. 
This will be used to pull your grading feedback back from the grader.
To pull changes, you'll typically use the command:

\begin{lstlisting}[language=bash]
git pull <remote> <branch>
\end{lstlisting}

Here, "remote" and "branch" have the same meanings as in the "git push" command.

Good practice is to pull just before any time you push. 

\begin{figure}[H]
    \centering
    \begin{tikzpicture}
        % Define styles
        \tikzstyle{Box}=[rectangle,draw=black!35,font=\sffamily\footnotesize,align=center,minimum height=.8cm,minimum width=2.75cm]
        \tikzstyle{Command}=[rectangle,draw=none,fill=none,font=\sffamily\footnotesize,align=center]
        % Rectangles
        \node[Box,fill=black!10!white,text=black!80!white] at (0,1.5) (B1) {Online Repository};
        \node[Box,fill=black!10!white,text=black!80!white] at (0,-1.5) (B2) {Computer};
        % Words (commands)
        \node[Command] at (-2.5,0) (C1) {\lif{git push origin main}};
        \node[Command] at (2.5,0) (C2) {\lif{git pull origin main}};
        % Lines
        \foreach \o/\i/\a/\b in {0/90/B1/C2,180/270/B2/C1} \draw[help lines,line width=.75pt,shorten >=-.1cm] (\a) to [out=\o,in=\i] (\b);
        % Arrows
        \foreach \o/\i/\a/\b in {270/0/C2/B2,90/180/C1/B1} \draw[->,>=stealth',help lines,line width=.75pt,shorten <=-.1cm] (\a) to [out=\o,in=\i] (\b);
    \end{tikzpicture}
    \caption{Exchanging git commits between the repository and a local clone.}
\end{figure}

\item \emph{Origin and Main}
"Origin" and "main" are two terms you'll often see when working with Git.

"Origin" refers to the default remote repository where your code is stored. 
When you clone a repository, Git sets up a remote called "origin" that points to the URL of the original repository. 
You can push and pull changes to/from "origin" to collaborate with others.

"Main" is the name of the default branch in a Git repository. This is the branch where the main line of development occurs, and it's typically the branch you'll be working on most of the time. 
However, depending on the repository, the default branch may be named something else (e.g., "master").
\end{itemize}
\begin{table}[H]
    \begin{tabular}{l|l}
        Command & Explanation \\ \hline
        \li{git status} & Display the staging area and untracked changes. \\
        \li{git pull origin main} & Pull changes from the online repository. \\
        \li{git push origin main} & Push changes to the online repository. \\
        \li{git add <filename(s)>} & Add a file or files to the staging area. \\
        \li{git add -u} & Add all modified, tracked files to the staging area. \\
        \li{git commit -m "<message>"} & Save the changes in the staging area with a given message. \\
        \li{git checkout -- <filename>} & Revert changes to an unstaged file since the last commit. \\
        \li{git reset HEAD -- <filename>} & Remove a file from the staging area. \\
        \li{git diff <filename>} & See the changes to an unstaged file since the last commit. \\
        \li{git diff --cached <filename>} & See the changes to a staged file since the last commit. \\
        \li{git config --local <option>} & Record your credentials (\li{user.name}, \li{user.email}, etc.). \\
    \end{tabular}
    \caption{Common git commands.}
\end{table}
\fi