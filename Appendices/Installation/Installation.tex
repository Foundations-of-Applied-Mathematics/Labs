\lab{Installing and Managing Python}{Installing and Managing Python}
\label{pythoninstall}

\objective{One of the great advantages of Python is its lack of overhead: it is relatively easy to download, install, start up, and execute.
This appendix introduces tools for installing and updating specific packages and gives an overview of possible environments for working efficiently in Python.
}

\section*{Installing Python via Anaconda} % ===================================

% There are a wide variety of ways to install most of the Python packages required for these labs.

% We suggest students use Anaconda because it includes nearly all the packages we use.
% They also tend to update their packages more often.
% It is also much less intrusive on the host filesystem.


A Python \emph{distribution} is a single download containing everything needed to install and run Python, together with some common packages.
For this curriculum, we \textbf{strongly} recommend using the \emph{Anaconda} distribution to install Python.
Anaconda includes IPython, a few other tools for developing in Python, and a large selection of packages that are common to applied mathematics, numerical computing, and data science.
Anaconda is free and available for Windows, Mac, and Linux.

Follow these steps to install Anaconda.
\begin{enumerate}
\item Go to \url{http://www.continuum.io/downloads}.
\item Download the Python 3.6 graphical installer specific to your machine.
\item Open the downloaded file and proceed with the default configurations.
\end{enumerate}

\begin{warn}
This curriculum uses Python 3.6, \textbf{not} Python 2.7.
With the wrong version of Python, some example code within the labs may not execute as intended or result in an error.
\end{warn}

For help with installation of Anaconda, visit \url{https://docs.continuum.io/anaconda/install/}. This page contains links to detailed step by step installation instructions for each operating system, as well as information for updating and uninstalling Anaconda.

\section*{Managing Packages} % ================================================

A \emph{package manager} is a tool for installing or updating Python packages, which involves downloading the right source code files, placing those files in the correct location on the machine, and linking the files to the Python interpreter.
\textbf{Never} try to install a package without using a package manager.

\subsection*{Conda} % ---------------------------------------------------------

Some packages are not included in the default Anaconda download but can still be installed using \li{conda}, Anaconda's package manager.
See \url{http://docs.continuum.io/anaconda/pkg-docs.html} for the complete list.
When you need to update or install a package, \textbf{always} try using \li{conda} first.

\begin{table}[H] % Conda commands.
\centering
\begin{tabular}{l|l}
    Command & Description \\
    \hline
    \li{conda install package-name} & Install the specified package.\\
    \li{conda update package-name} & Update the specified package.\\
    \li{conda update conda} & Update \li{conda} itself.\\
    \li{conda update anaconda} & Update \textbf{all} packages included in Anaconda.\\
    \li{conda --<<help>>} & Display the documentation for \li{conda}.
\end{tabular}
\end{table}

For example, the following commands attempt to install and update \li{matplotlib}.

\begin{lstlisting}
$ conda update conda                # Make sure that conda is up to date.
$ conda install matplotlib          # Attempt to install matplotlib.
$ conda update matplotlib           # Attempt to update matplotlib.
\end{lstlisting}

See \url{https://conda.io/docs/using/pkgs.html} for more detailed examples.

\begin{info}
The best way to ensure a package has been installed correctly is to try importing it in IPython.
\end{info}

\begin{warn}
Be careful not to attempt to update a Python package while it is in use.
It is safest to update packages while the Python interpreter is not running.
\end{warn}

\subsection*{Pip} % -----------------------------------------------------------

The most generic Python package manager is called \li{pip}.
While it has a larger package list, \li{conda} is the cleaner and safer option.
Only use \li{pip} to manage packages that are not available through \li{conda}.

\begin{table}[H] % Pip commands.
\centering
\begin{tabular}{l|l}
    Command & Description \\
    \hline
    \li{pip install package-name} & Install the specified package.\\
    \li{pip install --upgrade package-name} & Update the specified package.\\
    \li{pip freeze} & Display the version number on all installed packages.\\
    \li{pip --<<help>>} & Display the documentation for \li{pip}.
\end{tabular}
\end{table}

See \url{https://pip.pypa.io/en/stable/user_guide/} for more complete documentation.

\section*{Workflows} % ========================================================

There are several different ways to write and execute programs in Python.
Try a variety of workflows to find what works best for you.

\subsection*{Text editor + Terminal}
The most basic way of developing in Python is to write the program in a text editor and then run it using either the Python or IPython interpreter in the terminal.

There are many different text editors available for code development. Many text editors are designed specifically for computer programming which contain features such as syntax highlighting and error detection, and are highly customizable.

Some popular editors include:
\begin{itemize}
\item Atom (\url{https://atom.io/})
\item Sublime Text (\url{https://www.sublimetext.com/})
\item Notepad++ (Windows)
\item TextWrangler (Mac)
\item Geany (Linux)
\item Vim
\item Emacs
\end{itemize}

Once Python code has been written in a text editor, it can be executed in the terminal or command line. This can be done directly by running
\begin{lstlisting}
$ python <filename>  # Do not include < >
\end{lstlisting}
or through the Python or IPython interpreter.
The Python or IPython interpreters are opened by running
\begin{lstlisting}
$ python
\end{lstlisting}
or
\begin{lstlisting}
$ ipython
\end{lstlisting}
in the terminal respectively.
In these environments you can directly execute Python expressions individually, or run entire Python files.

IPython is an enhanced version of Python that is more user-friendly and interactive.
It has many features that cater to productivity such as tab completion and object introspection.

\begin{info}
While Mac and Linux computers come with a bash terminal built in, Windows computers do not. Windows does come with \emph{Powershell} which works similarly. Python and IPython both work in Powershell, along with commands such as \li{conda} and \li{pip}.However, some commands in Powershell are different than their bash analogs, and some programs (such as git) do not work in Powershell. Windows users who want a bash terminal can download git bash at \url{https://git-for-windows.github.io/}.
\end{info}


\subsection*{Integrated Development Environments}
Integrated Development Environments (IDEs) provide a comprehensive environment with all the tools necessary for development combined into a single application.
Most IDEs have many tightly integrated tools that are easily accessible.


\subsubsection*{Eclipse with PyDev}

Eclipse is a general purpose IDE that supports many languages.
The PyDev plugin for Eclipse contains all the tools needed to start working in Python.
It includes a built-in debugger, and has a very nice code editor.
Eclipse with PyDev is available for Windows, Linux, and Mac OSX.

To download Eclipse, visit \url{http://www.eclipse.org/}.
 The PyDev pluggin can be installed through the Eclipse application or downloaded from the web.
For installation instructions, see \url{http://www.pydev.org/manual_101_install.html}.

\subsubsection*{Spyder}

Spyder: \url{http://code.google.com/p/spyderlib/}
Spyder is a Python IDE that comes included with Anaconda.
Spyder comes with built in Python and IPython consoles, as well as interactive testing and debugging features.

\subsection*{Jupyter Notebook}

The Jupyter Notebook (also known as IPython Notebook) is a browser-based interface for Python that comes included as part of the Anaconda Python Distribution. It has an interface similar to the IPython interpreter, except that input is stored in cells and can be modified and re-evaluated as desired.

To begin using Jupyter Notebook, run the following command into the terminal:
\begin{lstlisting}
$ jupyter notebook
\end{lstlisting}

This will open your file system in a web browser in the Jupyter framework. To create a Jupyter Notebook, click the \emph{New} drop down menu and choose ``Python 3" under the heading \emph{Notebooks}. A new tab will open with a new Jupyter Notebook.

Jupyter Notebooks differ from different forms of Python development in that notebook files contain not only the raw Python code, but also formatting information. As such, Juptyer notebook files cannot be run in any other development environment. They also have a different file extention (\texttt{.ipynb}) rather than the standard Python extension (\texttt{.py})

Jupyter Notebooks also supports Markdown, which is a text formatting engine, LaTeX formatting, and embedding images. This makes Jupyter Notebooks ideal for presenting code.
