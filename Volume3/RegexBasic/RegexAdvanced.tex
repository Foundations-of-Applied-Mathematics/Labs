% Needs serious work; reevaluate whether outline is good

% % % LAB OUTLINE
% Groups/Capturing
%     Non capturing groups
%     replace fxnality
%         re.sub
%     named groups
% Greedy vs. Lazy Repetition
% Backreferences
% Anchors (\b, explain ^ and $ more)
% 4 types of Lookaround
%     talk about whether it's supported in every

% Summary table of lab 2?
% % %

\lab{Algorithms}{Advanced Regular Expressions}{Advanced Regular Expressions}
\objective{TODO}
\label{lab:Regex_Advanced}

\section*{Greedy vs. Lazy Repetition}

In the previous lab on regular expressions, we said that the metacharacters \li{'*'}, \li{'+'}, and \li{"{}"} consumed as much text as possible. These are called \emph{greedy} operators. \emph{Lazy} operators, on the other hand, consumes the smallest amount necessary. The lazy operators are formed by putting a question mark after the analogous greedy operator; i.e., \li{'*?'}, \li{'+?'}, and \li{'{}?'}.
For example:
% This doesn't work as expected for some reason
\begin{lstlisting}
star_greedy = re.compile(r"a*")
plus_greedy = re.compile(r"a+")
curly_greedy = re.compile(r"a{2,}")
star_lazy = re.compile(r"a*?")
plus_lazy = re.compile(r"a+?")
curly_lazy = re.compile(r"a{2,}?")
get_matches = lambda regex: [("The match with '" + "a"*i + "' is '" + regex.search("a"*i).string + "'.") for i in range(4)]
get_matches(star_greedy)
get_matches(plus_greedy)
get_matches(curly_greedy)
get_matches(star_lazy)
get_matches(plus_lazy)
get_matches(curly_lazy)
\end{lstlisting}




