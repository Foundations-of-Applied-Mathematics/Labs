\pdfminorversion=4
\documentclass[opener-c,labs,red,nociteref]{HJnewsiambook}

% See command.tex for all package imports, environments, and special commands.
% -----------------------------------------------------------------------------
% command.tex
% This file contains the various package imports, environment declarations,
% and other miscellaneous commands.
% -----------------------------------------------------------------------------

% \usepackage{pkgloader}            % Use this to resolve package dependencies
\usepackage{afterpage}
\usepackage{algorithmicx}
\usepackage{amsmath, amsfonts, amscd, amssymb}
\usepackage{appendix}
\usepackage{array}
% \usepackage{bbm}
\usepackage{bigstrut}
\usepackage{blkarray}
\usepackage{color}
\usepackage{colortbl}
\usepackage{epsfig}
\usepackage{float}
\usepackage{framed}
\usepackage{gensymb}
\usepackage{graphicx}
\usepackage{hyperref}
\usepackage{import}
\usepackage{listings}
\usepackage{mathrsfs}
\usepackage{mathtools}
\usepackage{makeidx}
\usepackage{multicol}
\usepackage{multirow}
\usepackage{paralist}
\usepackage{relsize}
\usepackage{subcaption}
\usepackage{textcomp}
\usepackage{verbatim}
\usepackage{tikz}
\usepackage{xparse}
\usepackage{xcolor}
\usepackage{url}

\usepackage[plain]{algorithm}
\usepackage[noend]{algpseudocode}
% \usepackage[style=alphabetic,refsection=chapter,backref=true,backend=bibtex]{biblatex}
\usepackage[customcolors]{hf-tikz}
\usepackage[framemethod=tikz]{mdframed}

% \usepackage{caption}
% \usepackage{subcaption}
% \usepackage{textcomp}
%\input{macros}

% \LoadPackagesNow                  % Use this to resolve package dependencies

% Tikzpicture tools
\usetikzlibrary{arrows, automata, backgrounds, calendar, chains, decorations,
    matrix, mindmap, patterns, petri, positioning, shadows, shapes.geometric,
    trees, shapes, decorations.pathreplacing}
\usetikzlibrary{circuits.ee.IEC}

% Document settings ===========================================================

% Colors
\definecolor{red}{cmyk}{0, 1.00, 0.62,0}
\definecolor{blue}{cmyk}{1.00, .34, .0 .02 }    % blue
\definecolor{green}{cmyk}{0.7, 0, 1.0, 0.09 }   % greenish
\definecolor{yellow}{cmyk}{0, 0.16, 1.0, 0}     % yellow
\definecolor{gray}{cmyk}{0, 0, 0, 0.65}         % gray
\definecolor{purple}{cmyk}{.333, .867, 0, .059}

\hfsetfillcolor{red!02}
\hfsetbordercolor{red}

% Set lengths that are pleasing for screen display.
\setlength{\paperheight}{11in}
\setlength{\paperwidth}{8.5in}

% Set paragraph skips and line spacing
\linespread{1.05}
\setlength{\parskip} {2pt plus1pt minus1pt}

\makeatletter

% Make all floats centered
\g@addto@macro\@floatboxreset\centering

% Reset footnote counter every chapter
\@addtoreset{footnote}{chapter}
\makeatother

\floatstyle{ruled}
\restylefloat{algorithm}

\makeatletter
\Hy@AtBeginDocument{
    \def\@pdfborder{0 0 1}
    \def\@pdfborderstyle{/S/U/W 1}
}
\makeatother

% Make margins the same on both sides (independent of odd or even page)
 \newlength{\marg}
 \setlength{\marg}{1.23in} %% Set the desired margin length here.
 \usepackage{marginnote}
 \usepackage[inner=1.0\marg,top=\marg,outer=1.0\marg, bottom=\marg, marginparsep = 0.25em, marginparwidth = .7\marg]{geometry}


% Lab Commands ================================================================


\renewcommand{\chaptername}{Lab}
\newcommand{\lab}[2]{\chapter[#2]{#1}}
\newcommand{\objective}[1]{{\bf Lab Objective: } \emph{#1} \bigskip}

%% Full line comments in the Algorithmic environment.
\algnewcommand{\LineComment}[1]{\State \(\triangleright\) #1}

\newcommand{\labdependencies}[1]{}

% Misc. Environments ==========================================================


\newcounter{problemnum}[chapter]
\newenvironment{problem}{\begin{mdframed}[style=problem]\begin{problemnum}}{\end{problemnum}\end{mdframed}}
\newtheoremup{problemnum}{Problem}

\newenvironment{problem*}{\begin{mdframed}[style=problem]\begin{problemnum*}}{\end{problemnum*}\end{mdframed}}
\newtheoremup{problemnum*}[problemnum]{*Problem}


% Colors ======================================================================


\colorlet{shadecolor}{blue!10}
%\definecolor{shadecolor}{RGB}{186, 207, 188}
\colorlet{warning}{red!20!}       %{RGB}{255, 188, 163}
\colorlet{warnline}{red}          %{RGB}{255, 15, 15}
\colorlet{information}{green!20}
\colorlet{infoline}{green}
%\definecolor{information}{RGB}{235, 255, 223}
%\definecolor{infoline}{RGB}{69, 163, 11}
\colorlet{codebase}{yellow!30!}
\colorlet{codekeyword}{blue}
\colorlet{codecomment}{green}
\colorlet{codestring}{red}
\colorlet{unittest}{purple!10}
\colorlet{unittestline}{purple!80}


% Frame environments ==========================================================


\mdfdefinestyle{problem}{backgroundcolor=shadecolor,
                        skipabove=10pt,
                        skipbelow=10pt
                        leftmargin=20pt,
                        rightmargin=20pt,
                        innertopmargin=10pt,
                        innerbottommargin=10pt,
                        innerleftmargin=10pt,
                        middlelinewidth=0pt,
                        everyline=true,
                        linecolor=blue,
                        linewidth=2pt}

\newmdenv[
  %roundcorner=10pt,
  skipabove=10pt
  skipbelow=10pt
  leftmargin=20pt,
  rightmargin=20pt,
  backgroundcolor=warning,
  innertopmargin=10pt,
  innerbottommargin=10pt,
  innerleftmargin=10pt,
  middlelinewidth=0pt,
  everyline=true,
  linecolor=warnline,
  linewidth=2pt,
  font=\normalfont\normalsize,
  frametitlefont=\large\bfseries,
  frametitleaboveskip=1em,
  frametitlerule=true,
  frametitle={\sc Achtung!}
]{warn}


\newmdenv[
  %roundcorner=10pt,
  skipabove=10pt,
  skipbelow=10pt,
  leftmargin=20pt,
  rightmargin=20pt,
  backgroundcolor=information,
  outerlinewidth=0pt,
  outerlinecolor=infoline,
  innertopmargin=10pt,
  innerbottommargin=10pt,
  innerleftmargin=10pt,
  middlelinewidth=0pt,
  everyline=true,
  linecolor=infoline,
  linewidth=2pt,
  font=\normalfont\normalsize,
  frametitlefont=\large\bfseries,
  frametitleaboveskip=1em,
  frametitlerule=true,
  frametitle={\sc Note}
]{info}

\newmdenv[
  %roundcorner=10pt,
  skipabove=10pt,
  skipbelow=10pt,
  leftmargin=20pt,
  rightmargin=20pt,
  backgroundcolor=unittest,
  outerlinewidth=0pt,
  outerlinecolor=unittestline,
  innertopmargin=10pt,
  innerbottommargin=10pt,
  innerleftmargin=10pt,
  middlelinewidth=0pt,
  everyline=true,
  linecolor=unittestline,
  linewidth=2pt,
  font=\normalfont\normalsize,
  frametitlefont=\large\bfseries,
  frametitleaboveskip=1em,
  frametitlerule=true,
  frametitle={\sc Unit Test}
]{unittest}


%% Listings Environments ======================================================


% Default Environment
\lstset{
  language=Python,
  backgroundcolor=\color{codebase},   %\color[RGB]{250, 245, 182},
  tabsize=4,
  basewidth=.5em,
  rulecolor=\color{yellow},           %\color{black},
  basicstyle=\normalsize\ttfamily,    % code text size
  upquote=true,
  columns=fixed,
  extendedchars=true,
  breaklines=true,
  prebreak = \raisebox{0ex}[0ex][0ex]{\ensuremath{\hookleftarrow}},
  frame=lrtb,
  xleftmargin=5pt,
  xrightmargin=5pt,
  framesep=4pt,
  framerule=2pt,
  showtabs=false,
  showspaces=false,
  showstringspaces=false,
  morestring=[s]{"""}{"""},
  morestring=[s]{'''}{'''},
  keywordstyle=\color{codekeyword},   %\color[RGB]{42, 161, 152},
  commentstyle=\color{codecomment},   %\color[RGB]{108, 153, 8},
  stringstyle=\color{codestring},     %\color[RGB]{189, 78, 98},
  title=\lstname,
  captionpos=b,
  abovecaptionskip=-5pt,
  belowcaptionskip=-5pt,
  moredelim=[is][\color{black}]{<<}{>>},
  moredelim=[is][\color{red}]{<r<}{>r>},
  moredelim=[is][\color{blue}]{<b<}{>b>},
  moredelim=[is][\color{green}]{<g<}{>g>},
  moredelim=[is][\color{purple}]{<p<}{>p>},
  morekeywords={assert, bytes, self, super, with, as, yield, True, False, None, NotImplemented, BaseException, Exception, AssertionError, AttributeError, ImportError, IndexError, KeyError, KeyboardInterrupt, MemoryError, NameError, NotImplementedError, OSError, OverflowError, RecursionError, RuntimeError, StopIteration, SyntaxError, IndentationError, TabError, StandardError, SystemError, SystemExit, TypeError, ValueError, ZeroDivisionError, IOError, Warning, RuntimeWarning, FileExistsError, FileNotFoundError,
    SELECT, FROM, AS, INNER, JOIN, LEFT, OUTER,
    CROSS, ON, WHERE, CASE, IF,
    MIN, MAX, SUM, AVG, COUNT,
    TEXT, REAL
  },
  deletekeywords={compile, format}
}

% \surroundwithmdframed[
%         hidealllines=true,
%         backgroundcolor=codebase,
%         innerleftmargin=-5pt,
%         innertopmargin=-1pt,
%         innerrightmargin=0pt,
%         innerbottommargin=-5pt]{lstlisting}

% Including source code from a file on disk
\lstdefinestyle{FromFile}{language=Python,
                          frame=single,
                          numbers=left,
                          numberstyle=\tiny,
                          stepnumber=2,
                          numbersep=7pt,
                          numberfirstline=true,
                          abovecaptionskip=2pt,
                          belowcaptionskip=2pt
                          }

% Shell I/O.  Avoids syntax highlighting
\lstdefinestyle{ShellOutput}{language=}
\lstdefinestyle{ShellInput}{language=}

%% Deprecated Environments (Replaced by Algorithmic package)
\lstdefinestyle{pseudo}{basicstyle=\rmfamily,
                        upquote=true,
                        keywordstyle=\color{black}\bfseries,
                        commentstyle=\color[rgb]{0.133,0.545,0.133},
                        stringstyle=\color[rgb]{0.627,0.126,0.941},
                        }

\newcommand{\pseudoli}[1]{\lstinline[style=pseudo]!#1!}
\newcommand{\li}[1]{\lstinline[prebreak=]!#1!}
\newcommand{\lif}[1]{\lstinline[basicstyle=\footnotesize\ttfamily,language=Python,prebreak=]!#1!} % for inline code in footnotes.
\newcommand{\lsql}[1]{\lstinline[language=SQL,prebreak=,
                                morekeywords={TEXT, REAL, IF}]!#1!}


% Special Math Characters =====================================================


\def\0{\mathbf{0}}
\def\a{\mathbf{a}}
\def\b{\mathbf{b}}
\def\c{\mathbf{c}}
\def\e{\mathbf{e}}
\def\f{\mathbf{f}}
\def\g{\mathbf{g}}
\def\p{\mathbf{p}}
\def\q{\mathbf{q}}
\def\u{\mathbf{u}}
\def\v{\mathbf{v}}
\def\w{\mathbf{w}}
\def\x{\mathbf{x}}
\def\y{\mathbf{y}}
\def\z{\mathbf{z}}
\def\subspace{\lhd}

\def\CalL{\mathcal{L}}
\def\CalO{\mathcal{O}}
\def\CalV{\mathcal{V}}
\def\CalU{\mathcal{U}}
\def\bU{{\bar{u}}}
\def\R{\Re e}
\def\I{\Im m}
\def\M{M_n}

\def\lvl#1{\multicolumn{1}{|c}{#1}} % Left  Vertical Line in array cell.
\def\rvl#1{\multicolumn{1}{c|}{#1}} % Right Vertical Line in array cell.

% Various other shortcuts =====================================================

\renewcommand{\epsilon}{\varepsilon}                    % curly epsilon
\newcommand{\argmax}{\mbox{argmax}}
\newcommand{\indicator}{\boldsymbol{1}}
% \newcommand{\indicator}{\mathbbm{1}} % characteristic (indicator) function
% ^Travis hates this, but it would be nice to have bbm instead.
\providecommand{\abs}[1]{\left\lvert#1\right\rvert}
\providecommand{\norm}[1]{\left\lVert#1\right\rVert}
\providecommand{\set}[1]{\lbrace#1\rbrace}
\providecommand{\setconstruct}[2]{\lbrace#1:#2\rbrace}
\providecommand{\Res}[1]{\underset{#1}{Res}}            % Residue
\newcommand{\trp}{^{\mathsf T}}                         % matrix transpose
\newcommand{\hrm}{^{\mathsf H}}                         % hermitian conjugate

\newcommand{\ipt}[2]{\langle #1,#2 \rangle}
\newcommand{\ip}{\int_{-\infty}^{+\infty}}

\renewcommand{\ker}[1]{\mathcal{N}(#1)}
\newcommand{\ran}[1]{\mathcal{R}(#1)}

% For making block arrays with correct bracket sizes
\newcommand\topstrut[1][0.8ex]{\setlength\bigstrutjot{#1}{\bigstrut[t]}}
\newcommand\botstrut[1][0.6ex]{\setlength\bigstrutjot{#1}{\bigstrut[b]}}

% These commands are specifically for use in the pseudocode environment.
% Load the xparse package to use these commands
\NewDocumentCommand\allocate{m+g}{                      % Empty array
  \IfNoValueTF{#2}
    {\mathrm{empty}(#1)}                                % 1 dimension
    {\mathrm{empty}(#1, #2)}                            % 2 dimensions
}

\NewDocumentCommand\zeros{m+g}{                         % Zero array
  \IfNoValueTF{#2}
    {\mathrm{zeros}(#1)}
    {\mathrm{zeros}(#1, #2)}%
}

\newcommand{\Id}[1]{\mathrm{Id}(#1)}                    % Identity array
\newcommand{\makecopy}[1]{\mathrm{copy}(#1)}            % Copy an array
\newcommand{\shape}[1]{\mathrm{shape}(#1)}
\newcommand{\size}[1]{\mathrm{size}(#1)}


% Math Operators ==============================================================

% Many of these are for use in the pseudocode environments.
\DeclareMathOperator{\sign}{sign}
\DeclareMathOperator{\sech}{sech}
\DeclareMathOperator{\Out}{Out}                         % Used in PageRank lab
\DeclareMathOperator{\In}{In}                           % Used in PageRank lab
\DeclareMathOperator\erf{erf}


\makeindex

\title{Volume 2\\ Algorithm Design and Optimization}
\author{Jeffrey Humpherys \& Tyler J.~Jarvis, managing editors}

\begin{document} % ============================================================

\newif\ifbyu
\byutrue
\byufalse % Create Public Labs; Comment for BYU Labs

\thispagestyle{empty} % Book cover and Front matter ---------------------------
\maketitle
\thispagestyle{empty}
\frontmatter

\begin{contributors}

% Professors
\contributor{B.~Barker}{Brigham Young University}
\contributor{E.~Evans}{Brigham Young University}
\contributor{R.~Evans}{Brigham Young University}
\contributor{J.~Grout}{Drake University}
\contributor{J.~Humpherys}{Brigham Young University}
\contributor{T.~Jarvis}{Brigham Young University}
\contributor{J.~Whitehead}{Brigham Young University}

% Students
\contributor{J.~Adams}{Brigham Young University}
\contributor{K.~Baldwin}{Brigham Young University}
\contributor{J.~Bejarano}{Brigham Young University}
\contributor{J.~Bennett}{Brigham Young University}
\contributor{A.~Berry}{Brigham Young University}
\contributor{Z.~Boyd}{Brigham Young University}
\contributor{M.~Brown}{Brigham Young University}
\contributor{A.~Carr}{Brigham Young University}
\contributor{C.~Carter}{Brigham Young University}
\contributor{S.~Carter}{Brigham Young University}
\contributor{T.~Christensen}{Brigham Young University}
\contributor{M.~Cook}{Brigham Young University}
\contributor{M.~Cutler}{Brigham Young University}
\contributor{R.~Dorff}{Brigham Young University}
\contributor{B.~Ehlert}{Brigham Young University}
\contributor{M.~Fabiano}{Brigham Young University}
\contributor{K.~Finlinson}{Brigham Young University}
\contributor{J.~Fisher}{Brigham Young University}
\contributor{R.~Flores}{Brigham Young University}
\contributor{R.~Fowers}{Brigham Young University}
\contributor{A.~Frandsen}{Brigham Young University}
\contributor{R.~Fuhriman}{Brigham Young University}
\contributor{T.~Gledhill}{Brigham Young University}
\contributor{S.~Giddens}{Brigham Young University}
\contributor{C.~Gigena}{Brigham Young University}
\contributor{M.~Graham}{Brigham Young University}
\contributor{F.~Glines}{Brigham Young University}
\contributor{C.~Glover}{Brigham Young University}
\contributor{M.~Goodwin}{Brigham Young University}
\contributor{R.~Grout}{Brigham Young University}
\contributor{D.~Grundvig}{Brigham Young University}
\contributor{S.~Halverson}{Brigham Young University}
\contributor{E.~Hannesson}{Brigham Young University}
\contributor{K.~Harmer}{Brigham Young University}
\contributor{J.~Henderson}{Brigham Young University}
\contributor{J.~Hendricks}{Brigham Young University}
\contributor{A.~Henriksen}{Brigham Young University}
\contributor{I.~Henriksen}{Brigham Young University}
\contributor{B.~Hepner}{Brigham Young University}
\contributor{C.~Hettinger}{Brigham Young University}
\contributor{S.~Horst}{Brigham Young University}
\contributor{R.~Howell}{Brigham Young University}
\contributor{E.~Ibarra-Campos}{Brigham Young University}
\contributor{K.~Jacobson}{Brigham Young University}
\contributor{R.~Jenkins}{Brigham Young University}
\contributor{J.~Larsen}{Brigham Young University}
\contributor{J.~Leete}{Brigham Young University}
\contributor{Q.~Leishman}{Brigham Young University}
\contributor{J.~Lytle}{Brigham Young University}
\contributor{E.~Manner}{Brigham Young University}
\contributor{M.~Matsushita}{Brigham Young University}
\contributor{R.~McMurray}{Brigham Young University}
\contributor{S.~McQuarrie}{Brigham Young University}
\contributor{E.~Mercer}{Brigham Young University}
\contributor{D.~Miller}{Brigham Young University}
\contributor{J.~Morrise}{Brigham Young University}
\contributor{M.~Morrise}{Brigham Young University}
\contributor{A.~Morrow}{Brigham Young University}
\contributor{R.~Murray}{Brigham Young University}
\contributor{J.~Nelson}{Brigham Young University}
\contributor{C.~Noorda}{Brigham Young University}
\contributor{A.~Oldroyd}{Brigham Young University}
\contributor{A.~Oveson}{Brigham Young University}
\contributor{E.~Parkinson}{Brigham Young University}
\contributor{M.~Probst}{Brigham Young University}
\contributor{M.~Proudfoot}{Brigham Young University}
\contributor{D.~Reber}{Brigham Young University}
\contributor{H.~Ringer}{Brigham Young University}
\contributor{C.~Robertson}{Brigham Young University}
\contributor{M.~Russell}{Brigham Young University}
\contributor{R.~Sandberg}{Brigham Young University}
\contributor{C.~Sawyer}{Brigham Young University}
\contributor{N.~Sill}{Brigham Young University}
\contributor{D.~Smith}{Brigham Young University}
\contributor{J.~Smith}{Brigham Young University}
\contributor{P.~Smith}{Brigham Young University}
\contributor{M.~Stauffer}{Brigham Young University}
\contributor{E.~Steadman}{Brigham Young University}
\contributor{J.~Stewart}{Brigham Young University}
\contributor{S.~Suggs}{Brigham Young University}
\contributor{A.~Tate}{Brigham Young University}
\contributor{T.~Thompson}{Brigham Young University}
\contributor{B.~Trendler}{Brigham Young University}
\contributor{M.~Victors}{Brigham Young University}
\contributor{E.~Walker}{Brigham Young University}
\contributor{J.~Webb}{Brigham Young University}
\contributor{R.~Webb}{Brigham Young University}
\contributor{J.~West}{Brigham Young University}
\contributor{R.~Wonnacott}{Brigham Young University}
\contributor{A.~Zaitzeff}{Brigham Young University}

\end{contributors}

% Use the following command to get all contributor usernames:
% git log --pretty=format:"%an" --since="1/3/2010" | sort | uniq


\begin{thepreface} % Preface --------------------------------------------------

This lab manual is designed to accompany the textbooks \emph{Foundations of Applied Mathematics Volume 2:  Algorithms, Approximation, and Optimization} by Humpherys and Jarvis.
The labs focus mainly on data structures, signal transforms, and numerical optimization, including applications to data science, signal processing, and machine learning.
The reader should be familiar with Python \cite{vanrossum2010python} and its NumPy \cite{oliphant2006guide,ascher2001numerical,oliphant2007python} and Matplotlib \cite{Hunter:2007} packages before attempting these labs.
See the Python Essentials manual for introductions to these topics.

\vfill
\copyright{This work is licensed under the Creative Commons Attribution 3.0 United States License.
You may copy, distribute, and display this copyrighted work only if you give credit to Dr.~J.~Humpherys.
All derivative works must include an attribution to Dr.~J.~Humpherys as the owner of this work as well as the web address to
\\
\centerline{\url{https://github.com/Foundations-of-Applied-Mathematics/Labs}}
\\
as the original source of this work.
\\
To view a copy of the Creative Commons Attribution 3.0 License, visit
\\
\centerline{\url{http://creativecommons.org/licenses/by/3.0/us/}}
or send a letter to Creative Commons, 171 Second Street, Suite 300, San Francisco, California, 94105, USA.}

\vfill
\centering\includegraphics[height=1.2cm]{by.pdf}
\vfill
\end{thepreface}

\setcounter{tocdepth}{1}
\tableofcontents

\mainmatter % LABS ============================================================

\ifbyu

\part{Labs} % BYU Volume 2 Labs ---------------------------------------------------
\subimport{./PythonEssentials/PythonIntro/}{PythonIntro}
\subimport{./PythonEssentials/NumpyIntro/}{NumpyIntro}
\subimport{./PythonEssentials/MatplotlibIntro/}{MatplotlibIntro}
\subimport{./PythonEssentials/UnitTesting/}{UnitTesting}
\subimport{./Volume2/BinaryTrees/}{BinaryTrees}
\subimport{./Volume2/NearestNeighbor/}{NearestNeighbor}
\subimport{./Volume2/BreadthFirstSearch/}{BreadthFirstSearch}
\subimport{./Volume2/Dijkstra/}{Dijkstra}
\subimport{./Volume2/MarkovChains/}{MarkovChains}
\subimport{./Volume2/Sampling/}{Sampling}
\subimport{./Volume2/FourierTransform/}{FourierTransform}
\subimport{./Volume2/ConvolutionFiltering/}{ConvolutionFiltering}

\subimport{./Volume2/Wavelets/}{Wavelets}
\subimport{./Volume2/PolynomialInterpolation/}{PolynomialInterpolation}
\subimport{./Volume2/GaussianQuadrature/}{GaussianQuadrature}
\subimport{./Volume2/OneD_Optimization/}{OneD_Optimization}
\subimport{./DataScienceEssentials/RegularExpressions/}{RegularExpressions}
\subimport{./Volume2/GradientMethods/}{GradientMethods}
\subimport{./Volume2/Simplex/}{Simplex}
\subimport{./Volume2/Gymnasium/}{Gymnasium}
\subimport{./Volume2/CVXPY_Intro/}{CVXPY_Intro}
\subimport{./Volume2/NMF/}{NMF}
\subimport{./Volume2/InteriorPoint_Linear/}{InteriorPoint_Linear}
\subimport{./Volume2/DynamicProgramming/}{DynamicProgramming}
\subimport{./Volume2/PolicyFunctionIteration/}{PolicyFunctionIteration}

\else 

\part{Labs} % Public Volume 2 Labs ---------------------------------------------------
\subimport{./Volume2/BinaryTrees/}{BinaryTrees}
\subimport{./Volume2/NearestNeighbor/}{NearestNeighbor}
\subimport{./Volume2/BreadthFirstSearch/}{BreadthFirstSearch}
\subimport{./Volume2/Dijkstra/}{Dijkstra}
\subimport{./Volume2/MarkovChains/}{MarkovChains}
\subimport{./Volume2/Sampling/}{Sampling}
\subimport{./Volume2/FourierTransform/}{FourierTransform}

\subimport{./Volume2/ConvolutionFiltering/}{ConvolutionFiltering}
\subimport{./Volume2/Wavelets/}{Wavelets}
\subimport{./Volume2/PolynomialInterpolation/}{PolynomialInterpolation}
\subimport{./Volume2/GaussianQuadrature/}{GaussianQuadrature}
\subimport{./Volume2/OneD_Optimization/}{OneD_Optimization}
\subimport{./Volume2/GradientMethods/}{GradientMethods}
\subimport{./Volume2/Simplex/}{Simplex}
\subimport{./Volume2/Gymnasium/}{Gymnasium}
\subimport{./Volume2/CVXPY_Intro/}{CVXPY_Intro}
\subimport{./Volume2/NMF/}{NMF}
\subimport{./Volume2/InteriorPoint_Linear/}{InteriorPoint_Linear}
\subimport{./Volume2/InteriorPoint_Quadratic/}{InteriorPoint_Quadratic}
\subimport{./Volume2/DynamicProgramming/}{DynamicProgramming}
\subimport{./Volume2/PolicyFunctionIteration/}{PolicyFunctionIteration}
%\subimport{./Volume2/LinkedLists/}{LinkedLists}

\fi

\part{Appendices}
\begin{appendices}
\subimport{./Appendices/NumpyVisualGuide/}{NumpyVisualGuide}
\subimport{./Appendices/MatplotlibCustomization/}{MatplotlibCustomization}
% \subimport{./Appendices/SklearnGuide/}{SklearnGuide}
\end{appendices}

% Bibliography
\bibliographystyle{alpha}
\bibliography{references}

\end{document}
